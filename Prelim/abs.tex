% !TeX root = ../thesis.tex
% ABSTRACT

The Fermi-Pasta-Ulam-Tsingou (FPUT) lattice became a system of great mathematical interest when it was observed that -- despite being a nonlinear dynamical system -- it exhibited a near-recurrence of its initial condition. This behavior was explained by observing that the Korteweg-de Vries (KdV) equation has soliton solutions and serves as a continuum limit for the FPUT. Much work has been done into analyzing the solitary wave solutions of the FPUT and the relationship between the lattice and its continuum limit. For certain potentials the modified KdV (mKdV) serves as a continuum for the FPUT. There has been little research done to examine how the defocusing mKdV can be used a modulation equation for the FPUT or how the kink solutions of the mKdV relate to solutions of the FPUT. This thesis first addresses the existence of kink-like solutions of the FPUT and shows that their profiles can be approximated by the profiles of the kink solutions of the mKdV. Next, it is shown that the defocusing mKdV can be used more widely as a modulation equation for small-amplitude, long-wavelength solutions of the FPUT lattice.
