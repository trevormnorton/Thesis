% !TeX root = ../thesis.tex
\addcontentsline{toc}{chapter}{Curriculum Vitae}

\thispagestyle{empty}

\begin{center}
	{\LARGE {\bf CURRICULUM VITAE}}\\
	\vspace{0.5in}
	{\large {\bf Trevor Norton}}
\end{center}


%%%%%%%%%%%%%%%%% CONTACT INFORMATION %%%%%%%%%%%%%%%%%
% Your email address, website, and Skype name are links to send email, open your website and add you on Skype. 

\begin{center}
	\begin{tabular}{l l}
		Email: nortontm@bu.edu & Phone: +1 (540) 355-6329 \\
	\end{tabular}


\vspace{0.25em}

%%%%%%%%%%%%%%%%% MAIN BODY %%%%%%%%%%%%%%%%%%%%%%%%%%%
% The main body is contained in a tabular environment. To move sections onto the next page, simply end the tabular environment and begin a new tabular environment.

\begin{tabbing}
	\hspace*{1cm}\=\hspace*{1cm}\=\hspace*{2.5cm}\= \kill
	\textbf{\large{Education}}  \\  
	\\
	\> {\textbf{Boston University}} \\
	\> \> \emph{Degree}: Doctor of Philosophy, Mathematics   \\
	\> \>\emph{Expected Date of Completion}: May 2023 \\
	\> \\
	\> {\textbf{Virginia Polytechnic Institutes and State University}} \\
	\> \> \emph{Degree}: Master of Science, Mathematics  \\
	\>\>\emph{Date Received}: May 2018 \\
	\> \\
	\> \> \emph{Degree}: Bachelor of Science, Applied Discrete Mathematics \\
	\> \> \emph{Minor}: Computer Science \\
	\> \> \emph{Date Received}: May 2015 \\
	\> \\
	\textbf{\large{Research}}    \\
	\\
	\> \textbf{{Kink-like Solutions for the FPUT Lattice and the mKdV as a}}\\
	\> \textbf{{Modulation Equation}} \\
	\> \> \emph{Description}: Doctoral Thesis \\
	\> \> \emph{Advisor}: C. Eugene Wayne \\
	\> \> \emph{Summary}: \= This thesis showed the existence of a kink-like solution to the \\
	\> \> \> FPUT and gave more general approximation  results for using \\
	\> \> \> the mKdV as a modulation equation for small-amplitude, \\
	\> \> \> long-wavelength solutions. \\
	\> \\
	\> \textbf{{Analyticity of Solutions to the Nonlinear Poisson Boltzmann}} \\ 
	\> \textbf{{Equation}} \\
	\> \> \emph{Description}: Research in collaboration with Mark Kon/Julio Castrillon \\
	\> \> \emph{Summary}: \= This research showed the analyticity of solutions of the\\
	\> \> \> nonlinear Poisson Boltzmann with respect to random \\
	\> \> \> perturbations of the domain. This has direct applications to \\
	\> \> \> Uncertainty Quantification. \\
	\> \\
	\> \textbf{{Galerkin Approximations of General Delay Differential Equations}} \\
	\> \textbf{{with Multiple Discrete or Distributed Delays}} \\
	\> \> \emph{Description}: Master's thesis \\
	\> \> \emph{Advisor}: Honghu Liu \\
	\> \> \emph{Summary}: \= Results are extended from a previous paper, in which a \\
	\> \> \> family of orthogonal polynomials are used to construct a  \\
	\> \> \> Galerkin approximation of a solution to a delay differential   \\
	\> \> \> equation (DDE). This method is generalized for DDEs of the  \\
	\> \> \> form \(\frac{\mathrm d x}{\mathrm d t} = ax(t) + bx(t-\tau) + F(x(t-\tau)).\) \\
	\\
	\> \\
	\> \textbf{{Combinatorial Curve Neighborhoods for the Affine Flag Manifold }}  \\
	\> \textbf{{of Type $A^1_1$}} \\
	\> \> \emph{Description}: Undergraduate research project\\
	\> \> \emph{Advisor}: Leonardo Mihalcea\\
	\> \> \emph{Summary}: \= Let \(X\) be the affine flag manifold of Lie type \(A^1_1\) and let \(D_\infty\)  \\ 
	\> \> \> denote the infinite dihedral group. The paper provides a  \\
	\> \> \> formula for the elements of the combinatorial curve  \\
	\> \> \> neighborhood for a given a fixed point \(u\in D_\infty\) and a degree \\
	\> \> \> \(\mathbf d = (d_0,d_1)\in \mathbb Z^2_{\geq 0}\).
	\\
	\\ \\
	\textbf{\large{Teaching}} \\
	\\
	\> Calculus I, Boston University \rpt[30]{.\,} \` Summer 2, 2022 \\
	\> Discrete Mathematics, Boston University \rpt[18]{.\,} \` Summer 2, 2020 \\
	\> Linear Algebra, Boston University \rpt[25]{.\,} \`Summer 2, 2019 \\
	\> Linear Algebra, Boston University \rpt[25]{.\,}  \`Summer 1, 2019 \\
	\\
	\textbf{\large{Work Experience}}  \\ 
	\> \\
	\> \textbf{Graduate Teaching Fellow} \\
	\> \> \emph{Location}: Boston University \\
	\> \> \emph{Dates}: September 2018 to May 2023 \\
	\> \>  \hspace{1em} \= $-$ \= Led discussion sections for undergraduate mathematics, statistics, \\
	\> \> \> \>  and computer science classes. \\ 
	\> \> \> $-$ Provided assistance to course instructors in implementing and \\
	\> \> \> \>grading student examinations. \\
	\> \> \> $-$ Tutored students one-on-one for various mathematics and \\
	\> \> \> \> statistics classes. \\
	\> \\
	\> \textbf{Graduate Research/Teaching Assistant} \\
	\> \> \emph{Location}: Virginia Polytechnic Institute and State University \\
	\> \> \emph{Dates}: August 2016 to May 2018 \\
	\> \>  \hspace{1em} \= $-$ \=Tutored undergraduate students in several freshman- and \\
	\> \> \> \> sophomore-level mathematics courses. \\
	\> \> \> $-$ Underwent teaching certification for university mathematics \\
	\> \> \> \> classes. \\
	\> \\ 
	\> \textbf{Software Engineer} \\
	\> \> \emph{Location}: Science Applications International Corporation (SAIC) \\
	\> \> \emph{Dates}: July 2015 to July 2016 \\
	\> \> \hspace{1em} \= $-$ \=Maintained and expanded the capabilities of a web application, \\
	\> \> \> \> which used the JavaServer Faces framework and an SQL database. \\
	\> \> \> $-$ Designed new features for the application in Java, using Hibernate \\
	\> \> \> \> to integrate with the database. \\
	\> \> \> $-$ Responsible for software maintenance across the application, which \\
	\> \> \> \> required knowledge of the entire system.\\
	\> \\
	\textbf{\large{Awards}}    \\ \\
	\> \textbf{Outstanding Applied Discrete Mathematics Senior} \\
	\> May, 2015 \\
\end{tabbing}
\end{center}