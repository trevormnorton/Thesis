% !TeX root = ../thesis.tex
% This file contains all the necessary setup and commands to create
% the preliminary pages according to the buthesis.sty option.

\title{A BU Thesis Latex Template}

\author{Trevor Norton}

% Type of document prepared for this degree:
%   1 = Master of Science thesis,
%   2 = Doctor of Philisophy dissertation.
%   3 = Master of Science thesis and Doctor of Philisophy dissertation.
\degree=2

\prevdegrees{B.S., Virginia Polytechnic Institute and State University, 2015\\
	M.S., Virginia Polytechnic Institute and State University, 2018}

\department{Department of Mathematics and Statistics}

% Degree year is the year the diploma is expected, and defense year is
% the year the dissertation is written up and defended. Often, these
% will be the same, except for January graduation, when your defense
% will be in the fall of year X, and your graduation will be in
% January of year X+1
\defenseyear{2022}
\degreeyear{2022}

% For each reader, specify appropriate label {First, Second, Third},
% then name, and title. IMPORTANT: The title should be:
%   "Professor of Electrical and Computer Engineering",
% or similar, but it MUST NOT be:
%   Professor, Department of Electrical and Computer Engineering"
% or you will be asked to reprint and get new signatures.
% Warning: If you have more than five readers you are out of luck,
% because it will overflow to a new page. You may try to put part of
% the title in with the name.
\reader{First}{First M. Last, PhD}{Professor of Electrical and Computer Engineering}
\reader{Second}{First M. Last}{Associate Professor of \ldots}
\reader{Third}{First M. Last}{Assistant Professor of \ldots}

% The Major Professor is the same as the first reader, but must be
% specified again for the abstract page. Up to 4 Major Professors
% (advisors) can be defined. 
\numadvisors=2
\majorprof{First M. Last, PhD}{{Professor of Electrical and Computer Engineering\\Secondary appointment}}
\majorprofb{First M. Last, PhD}{{Professor of computer Science}}
%\majorprofc{First M. Last, PhD}{{Professor of Astronomy}}
%\majorprofd{First M. Last, PhD}{{Professor of Biomedical Engineering}}

%%%%%%%%%%%%%%%%%%%%%%%%%%%%%%%%%%%%%%%%%%%%%%%%%%%%%%%%%%%%%%%%  

%                       PRELIMINARY PAGES
% According to the BU guide the preliminary pages consist of:
% title, copyright (optional), approval,  acknowledgments (opt.),
% abstract, preface (opt.), Table of contents, List of tables (if
% any), List of illustrations (if any). The \tableofcontents,
% \listoffigures, and \listoftables commands can be used in the
% appropriate places. For other things like preface, do it manually
% with something like \newpage\section*{Preface}.

% This is an additional page to print a boxed-in title, author name and
% degree statement so that they are visible through the opening in BU
% covers used for reports. This makes a nicely bound copy. Uncomment only
% if you are printing a hardcopy for such covers. Leave commented out
% when producing PDF for library submission.
%\buecethesistitleboxpage

% Make the titlepage based on the above information.  If you need
% something special and can't use the standard form, you can specify
% the exact text of the titlepage yourself.  Put it in a titlepage
% environment and leave blank lines where you want vertical space.
% The spaces will be adjusted to fill the entire page.
\maketitle
\cleardoublepage

% The copyright page is blank except for the notice at the bottom. You
% must provide your name in capitals.
\copyrightpage
\cleardoublepage

% Now include the approval page based on the readers information
\approvalpage
\cleardoublepage

% Here goes your favorite quote. This page is optional.
\newpage
%\thispagestyle{empty}
\phantom{.}
\vspace{4in}

\begin{singlespace}
\begin{quote}
  \textit{Facilis descensus Averni;}\\
  \textit{Noctes atque dies patet atri janua Ditis;}\\*
  \textit{Sed revocare gradum, superasque evadere ad auras,}\\
  \textit{Hoc opus, hic labor est.}\hfill{Virgil (from Don's thesis!)}
\end{quote}
\end{singlespace}

% \vspace{0.7in}
%
% \noindent
% [The descent to Avernus is easy; the gate of Pluto stands open night
% and day; but to retrace one's steps and return to the upper air, that
% is the toil, that the difficulty.]

\cleardoublepage

% The acknowledgment page should go here. Use something like
% \newpage\section*{Acknowledgments} followed by your text.
\newpage
\section*{\centerline{Acknowledgments}}
% !TeX root = ../thesis.tex

I would like to thank my advisor, Gene, for all his help with this thesis. He originally suggested the topic of research, and since then he has been giving guidance and feedback on my work. This thesis would not have been possible without his help, and I greatly appreciated having him as an advisor.

%Thank the committee 

%Thank Julio and Mark

%Thank my parents

%Thank Stuart

This research was supported in part by the US National Science Foundation through grant DMS-1813384 whose assistance is gratefully acknowledged.


%\vskip 1in

%\noindent
%Janusz Konrad\\
%Professor\\
%ECE Department
\cleardoublepage

% The abstractpage environment sets up everything on the page except
% the text itself.  The title and other header material are put at the
% top of the page, and the supervisors are listed at the bottom.  A
% new page is begun both before and after.  Of course, an abstract may
% be more than one page itself.  If you need more control over the
% format of the page, you can use the abstract environment, which puts
% the word "Abstract" at the beginning and single spaces its text.

\begin{abstractpage}
% !TeX root = ../thesis.tex
% ABSTRACT

[This is where the text for the abstract will go]

\end{abstractpage}
\cleardoublepage

% Now you can include a preface. Again, use something like
% \newpage\section*{Preface} followed by your text

% Table of contents comes after preface
\tableofcontents
\cleardoublepage

% If you do not have tables, comment out the following lines
%\newpage
%\listoftables
%\cleardoublepage

% If you have figures, uncomment the following line
%\newpage
%\listoffigures
%\cleardoublepage

% List of Abbrevs is NOT optional (Martha Wellman likes all abbrevs listed)
\chapter*{List of Abbreviations}
\begin{center}
  \begin{tabular}{lll}
    \hspace*{2em} & \hspace*{1in} & \hspace*{4.5in} \\
    FPUT  & \dotfill & Fermi-Pasta-Ulam-Tsingou \\
    mKdV  & \dotfill & modified Korteweg-De Vries\\
  \end{tabular}
\end{center}
\cleardoublepage

% END OF THE PRELIMINARY PAGES

\newpage
\endofprelim
