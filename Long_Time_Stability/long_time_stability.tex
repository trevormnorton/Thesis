% !TeX root = ../thesis.tex
\chapter{Long-Time stability of small FPUT solitary waves}
\label{chp:long-time-stability}
\pagestyle{myheadings}

\section{Introduction}

As shown in earlier work, there exists a wave solution of the FPUT lattice whose profile is well approximated by that of the kink solution to the (defocusing) mKdV. We are now interested in studying the stability of this wave solution on the FPUT lattice. The equations of motion on the lattice are given by 
\begin{equation}\label{fput-lattice-equations}
	\ddot x_n = V'(x_{n+1} - x_n) - V'(x_n - x_{n-1}), \quad n \in \Z.
\end{equation}
where \(V\) is the interaction potential between neighboring particles and \(\dot{\hspace{0.5em}}\) denotes the derivative with respect to the time \(t\in \mathbb{R}\). \Cref{fput-lattice-equations} can be rewritten in the strain variables \(u_n := x_{n+1} - x_n\) as follows
\begin{equation}\label{fput-lattice-equations-strain-variables}
	\ddot{u}_n = V'(u_{n+1}) - 2 V'(u_n) + V'(u_{n-1}), \quad n \in \Z
\end{equation}
The moving wave solution in \cref{fput-lattice-equations} corresponds to a kink solution in \cref{fput-lattice-equations-strain-variables}.

For the case where \(V\) is of the form \(V(u) = \frac 1 2 u^2 + \frac{\epsilon^2}{p+1} u^{p+1}\) for \(p\geq 2\), the generalized KdV equation given by 
\begin{equation}\label{generalized-KdV}
	2 \partial_TW + \frac 1 {12} \partial_X^3 W + \partial_X(W^p) = 0,\quad X\in\R
\end{equation} 
serves as a modulation equation for solutions of \cref{fput-lattice-equations-strain-variables} \cite{bambusi2006metastability,friesecke1999solitary}. That is, for a local solution \(W\in C([-\tau_0,\tau_0],H^s(\R))\) of \cref{generalized-KdV} there exist positive constants \(\epsilon_0\) and \(C_0\) such that, for all \(\epsilon \in (0,\epsilon_0)\), when initial data \((u_{\mathrm{in} }, \dot u_{\mathrm{in}})\in\ell^2(\R)\) satisfy
\begin{equation}
	\|u_{\mathrm{in}} - W(\epsilon\cdot, 0) \|_{\ell^2} + \| \dot u_{\mathrm{in}} + \epsilon \partial_X W(\epsilon\cdot, 0) \|_{\ell^2} \leq \epsilon^{3/2},
\end{equation}
the unique solution to \cref{fput-lattice-equations-strain-variables} with initial data \((u_{\mathrm{in} }, \dot u_{\mathrm{in}})\) belongs to \(C^1([-\tau_0\epsilon^{-3}, \tau_0 \epsilon^{-3}];\allowbreak \ell^2(\Z))\) and satisfies 
\begin{multline}
	\|u(t) - W(\epsilon (\cdot - t),\epsilon^3 t)\|_{\ell^2(\Z)} + \| \dot u(t) + \epsilon \partial_X W(\epsilon(\cdot - t), \epsilon^3 t) \|_{\ell^2(\Z)} \leq C_0 \epsilon^{3/2}, \\ t\in[-\tau_0 \epsilon^{-3},\tau_0 \epsilon^{-3}].
\end{multline}
Furthermore, the approximation can also be extended to include counter-propagating solutions of the KdV in the case where \(p=2\) \cite{schneider2000counter}.

The KdV approximation was extended to longer time scales on the order of \(\epsilon^{-3}|\log(\epsilon)|\) by Khan and Pelinovsky in order to deduce the nonlinear metastability of small FPUT solitary waves from the orbital stability of the corresponding KdV solitary waves \cite{khan2017long}.

We consider the FPUT with potential
\begin{equation}\label{truncated-potential}
	V(u) = \frac 1 2 u^2 - \frac{1} {24} u^4.
\end{equation}
We will introduce an ansatz that solutions of the FPUT with this potential can be well-approximated by counter-propagating solutions of mKdV equations.

\section{Counter-Propagating Waves Ansatz}

%Write out counter-propagting ansatz for solution 
We make the assumption that solutions of \cref{fput-lattice-equations-strain-variables} can be expressed as a sum of two counter-propagating small-amplitude waves, i.e., 
\begin{equation}\label{ansatz}
	u_n(t) \approx \epsilon f(\epsilon(n+t), \epsilon^3t) + \epsilon g(\epsilon(n-ct), \epsilon^3 t) + \epsilon^3\phi(\epsilon n, \epsilon t)
\end{equation}
where we allow \(f\) to have a fixed non-zero limit, \(f_\infty\), at positive infinity and \(\phi\) captures the interaction effects between \(f\) and \(g\). The wave speed of \(g\) is given by
\begin{equation}
	c = c(\epsilon, f_\infty) = 1 - \frac{\epsilon^2 f_\infty^2}{4}.
\end{equation}
Plugging in the ansatz in \cref{ansatz} back into \cref{fput-lattice-equations-strain-variables} and doing formal calculations gives that the approximation holds up to order \(\epsilon^6\) terms if the the functions \(f\) and \(g\) satisfy
\begin{equation}\label{f-mKdV}
	2 \partial_2 f = - \frac 1 6 \partial_1(f^3) + \frac 1 {12} \partial_1^3 f
\end{equation}
and 
\begin{equation}\label{g-gKdV}
	-2 \partial_2 g = - \frac 1 6 \partial_1 (g^3 + 3f_\infty g^2) + \frac 1 {12}\partial_1^3 g,
\end{equation}
and \(\phi\) satisfies
\begin{equation}\label{phi-pde}
	\begin{aligned}
		\partial_2^2 \phi(\xi, \tau) = \partial_1^2\phi(\xi, \tau)- \frac 1 6 \partial_1^2 \big[& 3(f^2(\xi+\tau,\epsilon^2\tau)-f_\infty^2)g(\xi-c\tau,\epsilon^2\tau) \\&+ 3 (f(\xi+\tau,\epsilon^2\tau)-f_\infty)g^2(\xi-c\tau,\epsilon^2\tau) \big ]
	\end{aligned}
\end{equation}
Note that \cref{f-mKdV} is the defocusing mKdV equation and \cref{g-gKdV} is a type of generalized KdV equation.

A natural choice of function space for \(g\) is a Sobolev space like \(H^k(\R)\). However, for \(f\), we want to allow the possibility of the function approaching a non-zero limit at positive and negative infinity while also having sufficient regularity. 
\begin{defn}
	For \(k\in\N\), let \(\mcX^k(\R)\) be the Banach space 
	\begin{equation}
		\mcX^k(\R) := \{f \in L^\infty(\R) \mid f'\in H^{k-1}(\R)\}
	\end{equation}
	with norm
	\begin{equation}
		\| f \|_{\mcX^k(\R)} := \| f \|_{L^\infty(\R)} + \| f' \|_{H^{k-1}(\R)}.
	\end{equation}
\end{defn}
Then \(\mcX^k\) is the set of \(L^\infty\) functions which are \(k\) times weakly differentiable and whose derivatives are in \(L^2\). That this is a Banach space follows from the Banach space isomorphism
\begin{equation}
	\mcX^k(\R) \cong L^\infty(\R) \cap \dot H^1(R) \cap \dot H^k(\R),
\end{equation}
where \(\dot H^k(\R)\) denotes the homogeneous Sobolev spaces. For convenience, we let \(\mcX^0(\R)\) denote \(L^\infty(\R)\)

The space \(\mcX^k\) is a natural one for \(f\), and allows \(f\) to be kink solutions of \cref{f-mKdV}. We also have the following inequalities for products of functions in \(\mcX^k\) and \(H^k\) that will be useful. 

	
\begin{lem}
	For non-negative integers \(k\), there is a \(C>0\) such that
	\begin{equation}\label{prod_rule}
		\| fg \|_{H^k} \leq C \| f \|_{\mathcal X^k} \| g \|_{H^k}
	\end{equation}
	for any \(f\in \mcX^k(\mathbb R)\) and \(g \in H^k(\mathbb R)\).
\end{lem}

\begin{lem}
	For non-negative integers \(k\), there is a \(C>0\) such that
	\begin{equation}
		\| fg \|_{\mcX^k} \leq C \| f \|_{\mcX^k} \| h \|_{\mcX^k}
	\end{equation}
	for any \(f,g\in \mcX^k(\mathbb R)\).
\end{lem}
See \cref{lemma-appendix} for proofs.

However, for our main result, we require that \(\phi\), the term which captures the interaction effects, remains uniformly bounded for all time. Intuitively, if \(f\) and \(g\) localized, the inhomogeneous term in \cref{phi-pde} will quickly go to zero and \(\phi\) will no longer experience growth in time. Thus we require that \(f\) and \(g\) quickly decay to their respective limits at infinity. This is enforced by assuming the functions belong to appropriate weighted Banach spaces.

A suitable choice of space for \(g\) is the weighted Sobolev spaces \(H^k_n(\R)\). Here, \(H^k_n\) for \(k,n\in\mathbb N\cup \{0\}\)
\begin{equation}
	H^k_n(\R) := \{ g\in H^k(\R) \mid   g\langle \cdot \rangle^n \in H^k \}
\end{equation}
where \(\langle x \rangle = \sqrt{1+x^2}\). The norm on this space is
\begin{equation}
	\| g \|_{H^k_n(\R)} := \|  g \langle \cdot \rangle^n\|_{H^k(\R)}.
\end{equation}
This space has the useful property that if \(g \in H^k_n\), then its Fourier transform, \(\hat g \), is in \(H^n_k\) and 
\begin{equation}
	c \| \hat g \|_{H^n_k} \leq \| g \|_{H^k_n} \leq C \| \hat g \|_{H^n_k}
\end{equation}
for \(c,C>0\) and independent on \(g\).

We want an analogous space for \(f\), but allowing for non-zero limits at infinity. Let \(\langle\cdot \rangle_+ :\R \to \R\) be a smooth function such that
\begin{equation}
	\langle x \rangle_+ = \begin{cases} \langle x \rangle, & x>1 \\ 1, & x<0\end{cases}
\end{equation}
and \(\langle \cdot \rangle_+\) continued smoothly between \(0\) and \(1\) such that it is always greater than or equal to \(1\). Thus \(\langle \cdot \rangle_+\) is a function that only acts like \(\langle \cdot \rangle\) for positive numbers. The function \(\langle \cdot \rangle_-\) is similarly defined but for the negative numbers.

\begin{defn}
	Define \(\mcX^k_{n^+} (\R)\) to be the Banach space of functions where 
	\begin{equation}
		\mcX^k_{n^+} (\R) := \{ f \in \mcX^k(\R) \mid \lim_{x\to\infty} f(x) = f_\infty\text{ and } (f-f_\infty)\langle\cdot\rangle_+^n \in \mcX^k(\R)\}
	\end{equation}
	with norm given by
	\begin{equation}
		\| f \|_{\mcX^k_{n^+}(\R)} := |f_\infty| + \|(f-f_\infty) \langle \cdot \rangle_+^n \|_{\mathcal X^k(\R)}
	\end{equation}
	Similarly, 
	\begin{equation}
		\mcX^k_{n^-} (\R) := \{ f \in \mcX^k(\R) \mid \lim_{x\to-\infty} f(x) = f_{-\infty}\text{ and } (f-f_{-\infty})\langle\cdot\rangle_-^n \in \mcX^k(\R)\}
	\end{equation}
	and 
	\begin{equation}
		\| f \|_{\mcX^k_{n^-}(\R)} := |f_{-\infty}| + \|(f-f_{-\infty}) \langle \cdot \rangle_-^n \|_{\mathcal X^k(\R)}
	\end{equation}
	Define \(\mcX^k_n(\R)\) to be the intersection of these Banach spaces. That is,
	\begin{equation}
		\mcX^k_n(\R) := \mcX^k_{n^+} (\R) \cap \mcX^k_{n^-} (\R), \quad \| f \|_{\mcX^k_{n} (\R)} := \|f\|_{\mcX^k_{n^+} (\R)} + \|f\|_{\mcX^k_{n^-} (\R)}.
	\end{equation}
\end{defn}
	That \(\mcX^k_{n^\pm}\) are Banach spaces follows from the fact that there exists a linear isomorphism between the Banach space \(\R\times \mcX^k\) and these spaces, which is given by
\begin{equation}
	(\alpha, f) \mapsto \alpha + f \langle \cdot \rangle^{-n}_{\pm}.
\end{equation}

%Prove that phi,psi remain bounded for all time
The definitions above prove that \(\phi\) remains bounded for all time. The idea behind the proof is similar to that of \cite[Lemma~3.1]{schneider2000counter}; if \(f\) and \(g\) are localized solutions, then the interaction terms of \cref{phi-pde} will decay quickly and so \(\phi\) will remain bounded. This decay can be quantified by the following lemma.
\begin{lem}
	For each \(k\geq 0\), there exists \(C> 0\) depending only on \(k\) such that 
	\begin{equation}\label{Ck-bound}
		\left \| \frac 1 {\langle \cdot +\tau\rangle_+^2 \langle \cdot - \tau \rangle^2} \right \|_{C^k} \leq C\, \sup_{x\in\mathbb R} \frac 1 {\langle x +\tau\rangle_+^2 \langle x - \tau \rangle^2}.
	\end{equation}
	Furthermore,
	\begin{equation}
		\int_0^\infty \sup_{x\in\mathbb R} \frac 1 {\langle x +\tau\rangle_+^2 \langle x - \tau \rangle^2}\, d\tau <\infty.
	\end{equation}
\end{lem}
See \cref{lemma-appendix} for proof. 

\section{Setup of Lattice Equations}

The scalar second-order differential equation \cref{fput-lattice-equations-strain-variables} with potential \(V\) given by \cref{truncated-potential} can be rewritten as the following first-order system:
\begin{equation}\label{first-order-lattice-eqns}
	\left\{\begin{aligned}\dot u _n &= q_{n+1} - q_n, \\
	\dot q_n &= u_n - u_{n-1} - \frac{1} 6 ( u_n^3 - u^3_{n-1}),\end{aligned} \right. \quad n \in \Z.
\end{equation}


%Define the ansatz and F and G

%Give assumptions on f,g and \dot{u}_n 

%Write the resulting equations for \mathcal{U} and \mathcal{Q}


\section{Preparatory Estimates}
To make estimates on the error terms \(\mathcal U\) and \(\mathcal Q\), we require an appropriate choice of \(W\). Based on the above discussion, \(W\) must be at least be a continuous solution in \(L^\infty(\R)\) space. Further regularity is required to make sense of \cref{p-epsilon-definition} as well as further derivatives of \(P_\epsilon\). So \(W\) must be bounded but have spatial derivatives that decay at infinity. The solution of \cref{modulating-eqn-defocusing-mkdv} must then be continuous in the following normed space.
\begin{defn}
	For \(s\geq 1\), the normed space \(\mathcal X_s\) is the set of functions
	\begin{equation}
		\mathcal X_s := \{u \in L^\infty(\R) : u'(x) \in H^{s-1}(\R)\}
	\end{equation}
	with a norm defined by 
	\begin{equation}
		\| u \|_{X_s} := \| u \|_{L^\infty(\R)} + \|u'\|_{H^{s-1}(\R)}.
	\end{equation}
\end{defn}
In fact, \(\mathcal X_s\) is a Banach space.
\begin{prop}
	For \(s\geq 1\), the normed vector space \(\mathcal X_s\) is a Banach space.
\end{prop}
\begin{proof}
	Suppose that \(\{u_n\}\subset \mathcal X_s\) is a Cauchy sequence. Then for any \(\epsilon > 0\) there exists \(N\in\mathbb N\) such that for any \(n,m\geq N\)
	\begin{equation}
		\| u_n - u_m \|_{\mathcal X_s} = \|u_n - u_m\|_{L^\infty(\R)} + \|u_n'-u_m'\|_{H^{s-1}(\R)} < \epsilon.
	\end{equation}
	Clearly, we have that \(\{u_n\}\) is a Cauchy sequence in \(L^\infty(\R)\) and \(\{u_n'\}\) is a Cauchy sequence in \(H^{s-1}(\R)\). Since \(L^\infty\) and \(H^{s-1}\) are Banach spaces, we have \(u \in L^\infty\) and \(v \in H^{s-1}\) such that 
	\begin{equation}
		\begin{aligned}
			u_n &\rightarrow u \quad \text{in } L^\infty(\R) \\
			u_n' &\rightarrow v \quad \text{in } H^{s-1}(\R).
		\end{aligned}
	\end{equation}
	To show that \(u_n \to u\) in \(\mathcal X_s\), we must demonstrate that \(v = u'\) (where \('\) denotes a distributional derivative which is \emph{a priori} defined). One can show that \(H^{s-1}(\R) \hookrightarrow L^2(\R)\): for \(w \in H^{s-1}(\R)\) we have
	\begin{equation}
	\begin{aligned}
		\|w \|_{L^2} &= \| \hat w \|_{L^2} \\
		&= \left( \int_\R |\hat w(\xi)|^2 \, d \xi \right)^{1/2} \\
		&\leq \left( \int_\R |\hat w(\xi)|^2 ( 1+ |\xi|^{2s})\, d \xi \right)^{1/2} \\
		&= \| w\|_{H^{s-1}}.
	\end{aligned}
	\end{equation}
	Thus we also have \(u_n' \to v\) in \(L^2(\R)\).
	
	Let \(\phi \in C_c^\infty(\R)\) be a test function. By \(L^2\) convergence, we have that 
	\begin{equation}
		\int_\R v \phi \, dx = \lim_{n\to\infty} \int_\R u_n' \phi \, dx.
	\end{equation}
	Also, by applying the dominated convergence theorem (and possibly taking a subsequence to get almost everywhere pointwise convergence), we have that
	\begin{equation}
		\int_\R u \phi' \, dx = \lim_{n\to\infty} \int_\R u_n \phi' \, dx.
	\end{equation}
	Therefore, we get that 
	\begin{equation}
	\begin{aligned}
		\int_\R v \phi \, dx &= \lim_{n\to\infty} \int_\R u_n' \phi \, dx \\
		&= \lim_{n\to\infty} - \int_\R u_n \phi' \, dx \\
		&= -\int_\R u\phi'\, dx .
	\end{aligned}
	\end{equation}
	Hence, \(u' = v\) and \(\mathcal X_s\) is a Banach space.
\end{proof}

For our case, we will need at least six spatial derivatives of \(W\) and so \(W\) should belong to \(C([-\tau_0,\tau_0];\mathcal X_6)\). Note that the kink solutions of \cref{modulating-eqn-defocusing-mkdv} are in this space; for instance, if \(W(X,T) = \phi(X)\) with 
\begin{equation}
	\phi(X) = \frac 1 {\sqrt 2} \tanh\left( \frac X {\sqrt{2}} \right),
\end{equation}
then \(W\) is a solution to the mKdV equation and \(W \in C(\R ; \mathcal X_s)\) for any \(s\geq 1\).

In order to prove long-time stability, we must get estimates of the \(\ell^2(\Z)\) norms of the residual terms and the nonlinearity. The following lemma proved in \cite{dumas2014justification} will be useful in bounding \(\ell^2(\Z)\) norms by Sobolev norms of \(W\).

\begin{lem}
	There exists \(C>0\) such that for all \(X \in H^1(\R)\) and \(\epsilon \in (0,1)\), \[\|x\|_{\ell^2} \leq C \epsilon^{-1/2} \|X\|_{H^1},\] where \(x_n := X(\epsilon n)\), \(n\in \mathbb Z\).
\end{lem}

We can then show the following estimates.

\begin{lem}\label{residual-nonlinear-estimates}
	Let \(W\in C([-\tau_0,\tau_0];\mathcal X_6)\) be a solution of the modified KdV equation \cref{modulating-eqn-defocusing-mkdv} and \(\tau_0>0\). Define
	\begin{equation}\label{delta-0-defn}
		\delta:= \sup_{\tau\in[-\tau_0,\tau_0]} \| W(\tau)\|_{\mathcal X_s}.
	\end{equation}
	There exists a positive \(\delta\)-independent constant \(C\) such that the residual and nonlinear terms satisfy
	\begin{equation}
		\| \mathrm{Res}^{(1)}(t) \|_{\ell^2} + \|\mathrm{Res}^{(2)}(t) \|_{\ell^2} \leq C \epsilon^{9/2} (\delta + \delta^5)
	\end{equation}
	and 
	\begin{equation}
		\|\mathcal R(W,\mathcal U)(t) \|_{\ell^2} \leq C \epsilon^2(\delta + \| \mathcal U(t) \|_{\ell^2}) \|\mathcal U (t)\|_{\ell^2}^2
	\end{equation}
	for every \(t \in [-\tau_0\epsilon^{-3}, \tau_0\epsilon^{-3}]\) and \(\epsilon \in (0,1).\)
\end{lem}
	
\begin{proof}
	Plugging \(P_\epsilon\) into \(\mathrm{Res}^{(1)}(t)\) and using the Taylor remainder theorem, we get that the terms of order \(\epsilon^4\) or lower cancel out and we are left with
	\begin{equation}
		\begin{aligned}
			\mathrm{Res}^{(1)}(t) = \epsilon^5 \Biggl[ &\quad\epsilon\left(c-1+\frac{\epsilon^2}{24}\right) \partial_X W(\epsilon(n-ct),\epsilon^3 t)\\
			&-\frac 1 {24} \int_0^1 \partial_X^5 W(\epsilon(n-ct+r), \epsilon^3 t)(1-r)^4 \, dr  \\ 
			&+ \frac 1 {12}  \int_0^1 \partial_X^5 W(\epsilon(n-ct+r), \epsilon^3 t)(1-r)^3 \, dr \\
			&- \frac 1 {16}  \int_0^1 \partial_X^5 W(\epsilon(n-ct+r), \epsilon^3 t)(1-r)^2 \, dr \\
			&+ \frac 1 {48}  \int_0^1 \partial_X^5 W(\epsilon(n-ct+r), \epsilon^3 t)(1-r) \, dr. \Biggr]
		\end{aligned}
	\end{equation}
	Each of these terms can be bounded by the \(\mathcal X_6\) norm. Thus we get that
	\begin{equation}
		\| \mathrm{Res}^{(1)}(t) \|_{\ell^2}  \leq C \epsilon^{9/2} \| W(\cdot, \epsilon t^3)\|_{\mathcal X_6} \leq C \epsilon^{9/2} \delta \quad \text{ for } t\in [-\tau_0\epsilon^{-3}, \tau_0\epsilon^{-3}].
	\end{equation}
	
	Plugging in \(P_\epsilon\) into \(\mathrm{Res}^{(2)}(t)\) similarly gives 
	\begin{equation}
	\begin{aligned}
		\mathrm{Res}^{(2)}(t) &= \epsilon\left(c-1 + \frac {\epsilon^2}{24} \right)\left( \partial_XP_\epsilon(\epsilon(n-ct),\epsilon^3 t) \right) \\
		&\qquad- \frac {\epsilon^3}{24} \left(\epsilon^2 \partial_X P_2(\epsilon(n-ct),\epsilon^3 t) + \epsilon^3 \partial_X P_3(\epsilon(n-ct),\epsilon^3 t) \right) \\
		&\qquad-\epsilon^3(\epsilon^2 \partial_TP_2(\epsilon(n-ct),\epsilon^3 t) + \epsilon^3 \partial_T P_3(\epsilon(n-ct),\epsilon^3 t)) \\
		&\qquad+ \frac{\epsilon^5}{24} \int_0^1 \partial_X^5 W(\epsilon(n-ct-r),\epsilon^3t)(r-1)^4 \, d r \\
		&\qquad- \frac {\epsilon^5}{12} \int_0^1 \partial^3_X(W^3)(\epsilon(n-ct-r), \epsilon^3 t)(r-1)^2 \, d r.
	\end{aligned}
	\end{equation}
	The terms with order less than \(\epsilon^5\) cancel out. The integral terms can easily be bounded by the \(\mathcal X_6\) norm For the first three terms, we can compute the expansion exactly:
	\begin{equation}
	\begin{aligned}
		\partial_X P_\epsilon &= - \partial_X^ {} W + \frac \epsilon 2 \partial_X^ 2 W + \frac{\epsilon^2} 4 W^2 \partial_X^ {} W - \frac{\epsilon^2} 8 \partial_X^ 3 W + \frac{\epsilon^3}{48} \partial_X^ 4 W - \frac{\epsilon^3}{24} \partial_X^ 2 (W^3) \\
		\partial_X^ {} P_2 &= \frac 1 4 W^2 \partial_X^ {} W - \frac 1 8 \partial_X^ 3 W \\
		\partial_X^ {} P_3 &= \frac 1 {48} \partial_X^ 4 W - \frac 1 4 W (\partial_X^ {} W)^2 - \frac 1 8 W^2 \partial_X^ 2 W \\
		\partial_T P_2 &= \frac{-1}{96} W^2 \partial_X^ {} W + \frac 1 {16} W^4 \partial_X^ {} W - \frac 1 {16} (\partial_X^ {} W)^3 - \frac 3 {16} W \partial_X^ {} W \partial_X^ 2 W \\
		&\qquad +\frac 1 {192} \partial_X^ 3 W - \frac 1 {24} W^2 \partial_X^ 3 W + \frac 1 {192} \partial_X^ 5 W \\
		\partial_T P_3 &= \frac 1 {192} W^2 \partial_X^ {} W - \frac 1 {32} W^4 \partial_X^ {} W + \frac 1 {96} W (\partial_X^ {} W)^2 - \frac 1 {16} W^3 (\partial_X^ {} W)^2 \\
		&\qquad+ \frac 1 {16} (\partial_X^ {} W)^2\partial_X^ 2 W + \frac 1 {32} W (\partial_X^ 2 W)^2 +\frac 1 {192} W^2 \partial_X^ 3 W + \frac 5 {96} W \partial_X^ {} W \partial_X^ 3 W \\
		&\qquad- \frac 1 {1152} \partial_X^ 4 W + \frac 1 {92} W^2 \partial_X^ 4 W - \frac 1 {1152} \partial_X^ 6 W
	\end{aligned}
	\end{equation}
	Notice that the \(L^2\) norm of each term can be bounded by a term of the form
	\begin{equation}
		\| W \|_{L^\infty}^k \| \partial_X W \|_{H^5}^\ell, \quad \text{ where } 1\leq k+\ell \leq 5.
	\end{equation}
	The above term can then in turn be bounded by \(C(\delta + \delta^5)\). Thus we get that 
	\begin{equation}
		\|\mathrm{Res}^{(2)}(t)\|_{\ell^2} = C\epsilon^{9/2} \left(\delta + \delta^5 \right)
	\end{equation}
	for \(t\in [-\tau_0\epsilon^{-3}, \tau\epsilon^{-3}]\).
	
	For the nonlinear term \(\mathcal R(W,\mathcal U)\), we immediately get that
	\begin{equation}
	\begin{aligned}
		\|\mathcal R(W,\mathcal U)(t) \|_{\ell^2} &\leq C \epsilon^2 \left[ \|W(\epsilon^3t)\|_{L^\infty} \|\mathcal U^2(t) \|_{\ell^2}  + \| \mathcal U^3(t) \|_{\ell^2} \right] \\
		&\leq C\epsilon^2\left[ \|W(\epsilon^3 t)\|_{L^\infty}  \|\mathcal U(t)\|_{\ell^\infty} \|\mathcal U(t)\|_{\ell^2} + \|\mathcal U(t)\|_{\ell^\infty}^2 \|\mathcal{U}(t)\|_{\ell^2}\right] \\
		&\leq C \epsilon^2 \left[ \delta \|\mathcal U(t) \|_{\ell^2}^2 + \|\mathcal U(t)\|_{\ell^2}^3\right].
	\end{aligned}
	\end{equation}
\end{proof}

The main result of this section will be proved using a Gr\"onwall type estimate using an energy function defined by 
\begin{equation}\label{energy-function}
	\mathcal E(t) := \frac 1 2 \sum_{n\in\Z} \mathcal Q_n^2(t) + \mathcal U_n^2(t) - \frac{\epsilon^2} 2 W^2(\epsilon(n-ct),\epsilon^3t)\mathcal U_n^2(t).
\end{equation}
The above will be nonnegative for \(W\) fixed and \(\epsilon \) sufficiently small.

\begin{lem}
	Let \(W\in C([-\tau_0,\tau_0],\mathcal X_6)\) be a solution to the mKdV equation \cref{modulating-eqn-defocusing-mkdv} and \(\tau_0 > 0\). Define \(\epsilon_0>0\) to be
	\begin{equation}\label{epsilon-0-defn}
		\epsilon_0 := \min\left\{1, \left( \sup_{\tau\in[-\tau_0, \tau_0]} \|W(\tau)\|_{L_\infty}\right)^{-1} \right\}.
	\end{equation}
	For every \(\epsilon \in (0,\epsilon_0)\) and for every local solution \((\mathcal U, \mathcal Q) \in C^1([-\tau_0\epsilon^{-3}, \tau_0\epsilon^{-3}], \ell^2(\mathbb Z))\) of \cref{error-lattice-eqns}, the energy-type quantity given in \cref{energy-function} is coercive with the bound
	\begin{equation}\label{coercive-bound}
		\|\mathcal Q(t) \|_{\ell^2}^2 + \| \mathcal U (t) \|_{\ell^2}^2 \leq  4 \mathcal E(t), \quad \text{for } t\in(-\tau_0\epsilon^{-3}, \tau_0\epsilon^{-3}).
	\end{equation}
	Moreover, when \(\delta\) is given by \cref{delta-0-defn}, there exists \(C> 0\) independent of \(\epsilon\) and \(\delta\) such that 
	\begin{equation}
		\left|\frac{d\mathcal E}{dt} \right| \leq C \mathcal E^{1/2}\left[ \epsilon^{9/2} (\delta + \delta^5)  + \epsilon^3(\delta^2 + \delta^4) \mathcal E^{1/2} + \epsilon^2(\delta + \mathcal{E}^{1/2})\mathcal E\right]
	\end{equation}
	for every \(t\in [-\tau_0\epsilon^{-3}, \tau_0\epsilon^{-3}]\) and \(\epsilon \in (0,\epsilon_0)\).
\end{lem}
\begin{proof}
	By the choice of \(\epsilon_0\), we have for \(\epsilon\in(0,\epsilon_0)\) that 
	\begin{equation}
		1 - \frac{\epsilon^2} 2 \| W \|_{L^\infty}^2 \geq \frac 12.
	\end{equation}
	Hence
	\begin{equation}
		\mathcal E(t) \geq \frac 1 2 \| \mathcal Q \|_{\ell^2}^2 + \frac 1 4 \| \mathcal U\|_{\ell^2}^2 \geq \frac 1 4\| \mathcal Q \|_{\ell^2}^2 + \frac 1 4 \| \mathcal U\|_{\ell^2}^2
	\end{equation}
	and \cref{coercive-bound} follows.
	
	Taking the time derivative of \(\mathcal E(t)\) and using that \((\mathcal U, \mathcal Q)\) solve \cref{error-lattice-eqns} gives us 
	\begin{equation}
	\begin{aligned}
		\frac{d \mathcal E}{dt} &= \sum_{n\in \mathbb{Z}} \Bigg[ \mathcal Q_n(t) \mathcal R_n(W,\mathcal U)(t) + \mathcal Q_n(t) \mathrm{Res}^{(2)}(t) \\
		&\qquad+ \mathcal U_n(t) \mathrm{Res}^{(1)}(t) \left( 1 - \frac{\epsilon^2} 2 W^2(\epsilon(n-ct),\epsilon^3 t) \right) \\
		&\qquad+\frac{\epsilon^3} 2 W(\epsilon(n-ct), \epsilon^3t) \mathcal U_n^2 (t) \left( c\partial_X W(\epsilon(n-ct),\epsilon^3 t) - \epsilon^2 \partial_TW(\epsilon(n-ct), \epsilon^3 t) \right) \Bigg]
	\end{aligned}	
	\end{equation}
	Using the Cauchy-Schwarz inequality, the estimates found in \cref{residual-nonlinear-estimates}, and coercivity of \(\mathcal E(t)\), we get that 
	\begin{equation}
	\begin{aligned}
		\left|\frac{d\mathcal E}{dt} \right| &\leq \|\mathcal Q\|_{\ell^2} \|\mathcal R(W,\mathcal U)\|_{\ell^2} + \|\mathcal Q\|_{\ell^2} \|\mathrm{Res}^{(2)}\|_{\ell^2} + \| \mathcal U \|_{\ell^2} \|\mathrm{Res}^{(1)}\|_{\ell^2} \\
		&\qquad+ \frac 1 2 \epsilon^3 \| W \|_{L^\infty} \|\mathcal U\|_{\ell^2}^2 ( c \|\partial_X W\|_{L^\infty} + \epsilon^2 \|\partial_T W\|_{L^\infty}) \\
		&\leq C \mathcal E^{1/2}\left[ \epsilon^{9/2} (\delta + \delta^5)  + \epsilon^3(\delta^2 + \delta^4) \mathcal E^{1/2} + \epsilon^2(\delta + \mathcal{E}^{1/2})\mathcal E\right]
	\end{aligned}
	\end{equation}
\end{proof}

\section{Proof of Long-Time Stability}

Now with the setup complete, the main result of this section can be shown. The result and proof are analogous to those of \cite[Thm.~1]{khan2017long}.

\begin{theorem}
	Let \(W \in C(\mathbb R; \mathcal X_6)\) be a global solution of the mKdV equation \cref{modulating-eqn-defocusing-mkdv} with \(\sup_{\tau\in \mathbb R} \| W(\tau)\|_{\mathcal X_6} \leq \delta\). For fixed \(r\in (0,1/2)\), there exist positive constants \(\epsilon_0\), \(C\), and \(K\) such that for all \(\epsilon \in (0,\epsilon_0)\), when initial data \((u_{\mathrm{in}}, q _{\mathrm{in}}) \in \ell^\infty(\Z)\times \ell^\infty(\Z)\) satisfy 
	\begin{equation}\label{initial-conditions}
		\|u_{\mathrm{in}} - W(\epsilon \cdot, 0) \|_{\ell^2} + \| q _{\mathrm{in}} + \epsilon \partial_XW(\epsilon \cdot, 0) \|_{\ell^2} \leq \epsilon^{3/2},
	\end{equation}
	the unique solution \((u,q)\) to the FPU equation \cref{first-order-lattice-eqns} belongs to 
	\begin{equation}
		C^1([-t_0(\epsilon), t_0(\epsilon)], \ell^\infty(\mathbb Z))
	\end{equation}
	with \(t_0(\epsilon):= r K^{-1} \epsilon^{-3} | \log (\epsilon) | \) and satisfies
	\begin{equation}
		\| u(t) - W(\epsilon(\cdot -ct), \epsilon^3 t) \|_{\ell^2} + \| q(t) + \epsilon \partial_X W(\epsilon (\cdot - ct), \epsilon^3 t) \|_{\ell^2} \leq C \epsilon^{3/2 - r}, \quad t\in[-t_0(\epsilon), t_0(\epsilon)].
	\end{equation}
\end{theorem}

\begin{proof}
	From the initial conditions satisfying \cref{initial-conditions}, we have at least local solutions to the error equations. That is, there is a unique local solution to \cref{error-evolution-equation} where \((\mathcal U, \mathcal Q) \in C^1((-t_0,t_0);\ell^2(\Z) \times \ell^2(\Z))\) for some \(t_0 > 0.\)
	
	Set \(\mathcal S := \mathcal E ^{1/2}\) where \(\mathcal E\) is defined in \cref{energy-function}. From the bound on the initial conditions in \cref{initial-conditions}, we get that \(\mathcal S(0) \leq C_0 \epsilon^{3/2}\) for some constant \(C_0 > 0\) and \(\epsilon_0\) chosen by \cref{epsilon-0-defn}. For fixed constants \(r\in(0,1/2)\), \(C> C_0\), and \(K > 0\), define the maximal continuation time by 
	\begin{equation}
		T_{C,K,r} := \sup \left\{T_0 \in (0, r K^{-1} \epsilon^{-3} |\log(\epsilon)|]: \mathcal S(t) \leq C \epsilon^{3/2 -r}, t\in [-T_0, T_0]\right\}.
	\end{equation} 
	We also define the maximal evolution time of the mKdV equation as \(\tau_0(\epsilon) = rK^{-1}|\log(\epsilon)|\). The goal is then to pick \(C\) and \(K\) so that \(T_{C,K,r} = \epsilon^{-3} \tau_0(\epsilon)\).
	
	We have that
	\begin{equation}
	\begin{aligned}
		\left | \frac d {dt} \mathcal S(t) \right | &= \frac 1 {2 \mathcal E ^{1/2}} \left | \frac d {dt} \mathcal E(t) \right| \\
		&\leq C_1(\delta + \delta^5) \epsilon^{9/2} + C_2 \epsilon^3\left[ (\delta^2 + \delta^4) + \epsilon^{-1}(\delta + \mathcal S) \mathcal S \right]\mathcal S
	\end{aligned}
	\end{equation}
	where \(C_1, C_2 > 0\) are independent of \(\delta\) and \(\epsilon\). While \(|t| \leq T_{C,K,r}\),
	\begin{equation}
		C_2 \left[ (\delta^2 + \delta^4) + \epsilon^{-1}(\delta + \mathcal S) \mathcal S \right] \leq C_2 \left[ (\delta^2 + \delta^4) + \epsilon^{-1}(\delta +  C\epsilon^{3/2-r}) C \epsilon^{3/2-r} \right],
	\end{equation}
	where the right-hand side is continuous in \(\epsilon \) for \(\epsilon \in [0,\epsilon_0]\). Thus the right-hand side can be uniformly bounded by a constant independent of \(\epsilon\). Choose \(K>0\) (dependent on \(C\)) sufficiently large so that 
	\begin{equation}\label{K-def}
		C_2 \left[ (\delta^2 + \delta^4) + \epsilon^{-1}(\delta +  C\epsilon^{3/2-r}) C \epsilon^{3/2-r} \right] \leq K.
	\end{equation}
	Hence, we can get that for \(t \in [-T_{C,K,r}, T_{C,K,r}]\) 
	\begin{equation}
	\begin{aligned}
		\frac d {dt} e^{-\epsilon^3 K t} \mathcal S(t) &= - \epsilon^3 K e^{-\epsilon^3 K t} \mathcal S  + e^{-\epsilon^3 K t} \frac d {dt} \mathcal S \\
		&\leq - \epsilon^3 K e^{-\epsilon^3 K t} \mathcal S  + e^{-\epsilon^3 K t}C_1(\delta + \delta^5) \epsilon^{9/2} \\
		&\qquad+ e^{-\epsilon^3 K t}C_2 \epsilon^3\left[ (\delta^2 + \delta^4) + \epsilon^{-1}(\delta + \mathcal S) \mathcal S \right]\mathcal S \\
		&\leq - \epsilon^3 K e^{-\epsilon^3 K t} \mathcal S  +  e^{-\epsilon^3 K t}C_1(\delta + \delta^5) \epsilon^{9/2} + \epsilon^3 K e^{-\epsilon^3 K t}\mathcal S \\
		&= e^{-\epsilon^3 K t}C_1(\delta + \delta^5) \epsilon^{9/2}.
	\end{aligned}
	\end{equation}
	Integrating gives
	\begin{equation} 
	\begin{aligned}
		\mathcal S(t) &\leq \left( \mathcal S(0) + K^{-1} C_1 (\delta+\delta^5) \epsilon^{3/2} \right) e^{\epsilon^3 K t} - \epsilon^{-3} K^{-1} C_1 (\delta + \delta^5) \\
		&\leq \left( \mathcal S(0) + K^{-1} C_1 (\delta+\delta^5) \epsilon^{3/2} \right) e^{\epsilon^3 K t} \\
		&\leq \left( \mathcal S(0) + K^{-1} C_1 (\delta+\delta^5) \epsilon^{3/2} \right) e^{ K \tau_0(\epsilon)} \\
		&\leq \left( C_0 + K^{-1} C_1 (\delta+\delta^5) \epsilon^{3/2} \right) \epsilon^{3/2 -r}
	\end{aligned}
	\end{equation}
	for \(t \in [-T_{C,K,r}, T_{C,K,r}]\), where the last line follows in part from the definition of \(\tau_0(\epsilon)\). Now choose \(C> C_0\) sufficiently large so that 
	\begin{equation}
		C_0 + K^{-1} C_1(\delta + \delta^5) \leq C.
	\end{equation}
	Note that our earlier choice of \(K\) can be enlarged so that \cref{K-def} still holds as well as the above inequality. Therefore, with these choices of \(C\) and \(K\), the maximal interval can be extended to \(T_{C,K,r} = \epsilon^{-3} \tau_0(\epsilon) \). 
\end{proof}