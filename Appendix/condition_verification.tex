% !TeX root = ../thesis.tex
\chapter{Fenichel Theory}\label{condition-verification}

\cite{wiggins1994normally}

\begin{equation}\label{generic_ode}
	\dot x = f(x), \quad x \in \R^n
\end{equation}

\begin{defn}
	Let \(\overline M = M \cup \partial M\) be a compact, connected \(C^r\) manifold with boundary contained in \(\R^n\). Then \(\overline M\) is said to be \emph{overflowing invariant} under \cref{generic_ode} if for every \(p \in \overline M\), \(\phi_t(p) \in \overline M\) for all \(t\leq 0\) and the vector field \cref{generic_ode} is pointing strictly outward on \(\partial M\).
\end{defn}

%Wiggins Thm. 4.5.1 originally Fenichel 1971
\begin{theorem}\label{unstable-manifold-fenichel}
	Suppose \(\dot x = f(x)\) is a \(C^r\) vector field on \(\R^n\), \(r\geq 1\). Let \(\overline M = M \cup \partial M\) be a \(C^r\), compact connected manifold with boundary overflowing invariant under the vector field \(f(x)\). Suppose \(\nu^s(p) < 1\), \(\lambda^u(p) < 1\), and \(\sigma^s(p) < \frac 1 r\) for all \(p\in M\). Then there exists a \(C^r\) overflowing invariant manifold \(W^u(\overline M)\) containing \(\overline M\) and tangent to \(h_u(N'^u_\epsilon )\) along \(\overline M\) with trajectories in \(W^u(\overline M)\) approaching \(\overline M\) as \(t\to -\infty\).
\end{theorem}

%Wiggins Thm 5.6.1
\begin{theorem}\label{foliation-of-unstable-manifold}
	Suppose \(\dot x = f(x) \) is a \(C^r\) vector field on \(\R^n\), \(r\geq 1\). Let \(\overline M = M \cup \partial M\) be a \(C^r\) compact connected manifold with boundary, overflowing invariant under the vector field \(f(x)\). Suppose \(\lambda^u(p)< 1\), \(\sigma^{cu}(p) < 1\), and \(\sigma^{su}(p) < 1\) for every \(p \in \overline M_1\). Then there exists a \(n-(s+u)\)-parameter family \(\mcF^u = \cup_{p\in M} f^u(p)\) of \(u\)-dimensional surfaces \(f^u(p)\) (with boundary) such that the following hold:
	\begin{enumerate}
		\item \(\mcF^u\) is a negatively invariant family, i.e., \(\phi_{-t} (f^u(p)) = f^u(\phi_{-t}(p))\) for any \(t\geq 0\) and \(p\in M\).
		\item The \(u\)-dimensional surfaces \(f^u(p)\) are \(C^r\).
		\item \(f^u(p)\) is tangent to \(h_u(N'^u_p)\) at \(p\).
		\item There exists \(C_u,\lambda_u > 0\) such that if \(q\in f^u(p)\), then \[| \phi_{-t}(q) - \phi_{-t}(p) | < C_u e^{-\lambda_u t}\] for any \(t\geq 0\).
		\item Suppose \(q\in f^u(p)\) and \(q'\in f^u(p')\). Then \[\frac{|\phi_{-t}(q) - \phi_{-t}(p) |}{| \phi_{-t}('q) - \phi_{-t}(p) |} \to 0 \quad \text{as } t\to\infty\] unless \(p = p'\).
		\item \(f^u(p) \cap f^u(p') = \emptyset \), unless \(p=p'\).
		\item If the hypotheses of the unstable manifold theorem hold, i.e., if additionally \(\nu^s(p) < 1\) and \(\sigma^s(p) < \frac 1 r\) for every \(p \in \overline M_1\), then the \(u\)-dimensional surfaces \(f^u(p)\) are \(C^r\) with respect to the basepoint \(p\).
		\item \(\mcF^u = W_{\mathrm{loc}}^u(M)\).
	\end{enumerate}
\end{theorem}


%See update for 10/14/2020 for more details about this.
The proofs for these lemmas should be similar to the proofs found in \cite{dieci1997lyapunov}.

\begin{lem}\label{lemma1}
	Let \(p \in M\) with \(\nu^s(p) < 1\). For \(c > \sigma^s(p)\), we have 
	\begin{equation}\label{eq1}
		\| A_t(p) \|\, \|B_t(p) \|^c \to 0 \quad \text{as } t\to\infty.
	\end{equation}
	Conversely, if \cref{eq1} holds for some \(c\in \R\), then \(c\geq \sigma^s(p).\)
\end{lem}

\begin{lem}\label{lemma2}
	For \(p\in M\) and \(s,t\geq 0\), we have the following:
	\begin{enumerate}[label=(\roman*)]
		\item \(A_{s+t}(p) = A_t(\phi_{-s}(p)) A_s(p)\) 
		\item \(B_{s+t}(p) = B_s(p) B_t(\phi_{-s}(p))\)
		\item \(C_{s+t}(p) = C_t(\phi_{-s}(p))C_s(p)\).
	\end{enumerate}
\end{lem}
\lyapunovbound*
\begin{proof}
	\emph{(i)} Let \(a\in\R\) such that \(\lambda^u(K) < a\). For each \(q\in K\) there is a \(\tau_q>0\) and an open, precompact neighborhood of \(q\), \(U_q\), such that \[\| C_{\tau_q}(q')\| < a^{\tau_q} \quad \text{for all } q'\in U_q.\] Then \(\{ U_q \}_{q\in K}\) is an open cover of \(K\), and so we can take a finite subcover \(\{U_i\}_{i=1}^m\) with associated \(\tau_q\) values denoted by \(\tau_i\) for \(i = 1,\ldots, m\). Let \(U = \bigcup_{i=1}^m U_i\) and assume without loss of generality that \(\tau_1\leq \tau_2\leq\cdots\leq \tau_m\). Since \(\lambda^u(p)\) is constant along trajectories and \(\phi_{-t}(p) \to K\) as \(t\to\infty\), we can assume that \(\phi_{-t}(p) \in U\) for all \(t\geq 0.\) 
	
	We can now break up the orbit of \(\phi_{-t}(p)\) into discrete times to keep track of which \(U_i\) the orbit lies in. We shall do this inductively. Set \(t_0 = 0\). Then \(\phi_{-t_0}(p) = p \in U_{i_0}\) for some index \(i_0\in \{1,2,\ldots, m\}\). Then we can define \(t_1 = t_0 + \tau_{i_0}\) and again we have \(\phi_{-t_1}(p) \in U_{i_1}\) for some index \(i_1\). We can continue this process. Suppose we have \(t_k\) and \(\tau_{i_k}\). Then
	\begin{equation*}
		t_{k+1} = t_k + \tau_{i_k}, \qquad \phi_{-t_{k+1}}(p) \in U_{i_{k+1}},
	\end{equation*}
	and so we have \(t_{k+1}\) and \(\tau_{i_{k+1}}\) defined. Note that \(t_{k+1} - t_k = \tau_{i_k} \leq \tau_{m}\), so the distance between times does not grow too large. Furthermore, we also have \(t_{k+1} - t_k = \tau_{i_k} \geq \tau_1\) and so \(t_k \to \infty\) as \(k\to\infty.\)
	
	Now suppose \(t>0\) is fixed and arbitrary. There is some \(\ell\) such that \(t_\ell \leq t < t_{\ell+1}\). Then there is some \(s < \tau_m\) such that 
	\begin{equation}
		\begin{aligned}
			t &= t_\ell + s \\
			&= \sum_{k=0}^{\ell - 1} \tau_{i_k} +s .
		\end{aligned}
	\end{equation}
	Using this decomposition of \(t\) along with \cref{lemma2}, we get that
	\begin{equation}
		\begin{aligned}
			C_t(p) &= C_{t_\ell+s}(p) \\
			&= C_s( \phi_{-t_\ell}(p) ) C_{t_\ell} (p) \\
			&= C_s( \phi_{-t_\ell}(p) ) C_{\tau_{i_{\ell-1}}}(\phi_{-t_{\ell-1}}(p)) C_{\tau_{i_{\ell-2}}}(\phi_{-t_{\ell-2}}(p)) \cdots C_{\tau_{i_0}}(p).
		\end{aligned}
	\end{equation}
	Thus we have
	\begin{equation}
		\begin{aligned}
			\| C_t(p) \| &\leq \| C_s(\phi_{-t_\ell}(p)) \| a^{\tau_{i_{\ell -1}}} \cdot a^{\tau_{i_{\ell-2}}} \cdots a^{\tau_{i_0}} \\
			&= \| C_s(\phi_{-t_\ell}(p)) \| a^{t_\ell}.
		\end{aligned}
	\end{equation}
	Defining a constant \(C_1\) by 
	\begin{equation}
		C_1 = \max\{ a^{-s} \| C_s(q) \| : q\in \overline U , 0 \leq s \leq \tau_m\}
	\end{equation}
	we can write
	\begin{equation}
		\|C_t(p) \| \leq C_1 a^s a^{t_\ell} = C_1 a^t.
	\end{equation}
	Since this \(C_1\) is independent of \(t\), raising both sides to \(1/t\) and taking the limit as \(t\to\infty\) gives us that 
	\begin{equation}
		\limsup_{t \to \infty} \| C_t(p) \| ^{1/t} \leq a,
	\end{equation}
	and so \(\lambda^u(p) \leq a\) for each \(a > \lambda^u(K).\) This proves \(\lambda^u(p) \leq \lambda^u(K)\).
	
	\emph{(ii)} We follow a similar argument for \(\nu^s(p)\). Let \(a\in \R\) such that \(\nu^s(K) < a\). We can find an open cover of \(K\) given by \(\{U_i\}_{i=1}^m\) (with each \(U_i\) precompact) and positive numbers \(\tau_1 \leq \tau_2\leq \cdots \leq \tau_m\) such that
	\begin{equation}
		\| B_{\tau_i}(q) \| < a^{\tau_i} \quad \text{for all } q\in U_i.
	\end{equation}
	The number \(\nu^s(p)\) is constant on orbits, so assume that \(\phi_{-t}(p) \in U := \cup_{i=1}^m U_i \) for all \(t\geq 0\). We can similarly construct the \(t_k\) and \(\tau_{i_k}\) inductively.
	
	Let \(t> 0\). Then there is an \(\ell\) such that \(t_\ell \leq t< t_{\ell +1}\), and we have \(0\leq s < \tau_m\) with 
	\begin{equation}
		\begin{aligned}
			t &= t_\ell + s \\
			&= \sum_{k=0}^{\ell - 1} \tau_{i_k} +s .
		\end{aligned}
	\end{equation}
	Thus
	\begin{equation}
		B_t(p) = B_{\tau_{i_0}}(p) B_{\tau_{i_1}}(\phi_{-t_1}(p)) \cdots B_{\tau_{i_{\ell-1}}}(\phi_{-t_{\ell-1}}(p)) B_s(\phi_{-t_\ell}(p)).
	\end{equation}
	We can then get 
	\begin{equation}
		\| B_t(p) \| \leq \| B_s(\phi_{-t_\ell}(p)) \| a^{t-s}.
	\end{equation}
	Defining a constant \(C_2\) by 
	\begin{equation}
		C_2 = \max\{ a^{-s} \| B_s(q) \| : q\in \overline U , 0 \leq s \leq \tau_m\}
	\end{equation}
	we can write
	\begin{equation}
		\|B_t(p) \| \leq C_2 a^s a^{t_\ell} = C_2 a^t.
	\end{equation}
	Taking limits gives us \(\nu^s(p) \leq a\) and thus \(\nu^s(p) \leq \nu^s(K).\)
	
	\emph{(iii)} Assume that \(\nu^p(K) < 1\). Let \(c > \sigma^s(K)\) be arbitrary. For each \(q\in K\), there is a \(\tau_q\) and a precompact, open neighborhood of \(q\), \(U_q\), such that 
	\begin{equation}
		\|A_{\tau_q} (q')\|\, \|B_{\tau_q}(q')\|^c \leq \frac 12, \quad \text{for all } q'\in U_q.
	\end{equation}
	We again take a finite subcover \(\{U_i\}_{i=1}^m\) with corresponding \(\tau_1\leq \tau_2 \leq \cdots \leq \tau_m.\) The number \(\sigma^s(p)\) is constant on orbits so assume that \(\phi_{-t}(p) \in U := \cup_{i=1}^mU_i\) for all \(t\geq 0.\) The \(t_k\) and \(\tau_{i_k}\) values are constructed the same way as in \emph{(i)}.
	
	For \(t> 0\), we have \(t_\ell\leq t < t_{\ell+1}\) and we can write \(t\) as
	\begin{equation}
		\begin{aligned}
			t &= t_\ell + s \\
			&= \sum_{k=0}^{\ell - 1} \tau_{i_k} +s 
		\end{aligned}
	\end{equation}
	with \(0\leq s <\tau_m.\) By our product formulas,
	\begin{equation}
		A_t(p) = A_s( \phi_{-t_\ell}(p) ) A_{\tau_{i_{\ell-1}}}(\phi_{-t_{\ell-1}}(p)) A_{\tau_{i_{\ell-2}}}(\phi_{-t_{\ell-2}}(p)) \cdots A_{\tau_{i_0}}(p)
	\end{equation}
	and 
	\begin{equation}
		B_t(p) = B_{\tau_{i_0}}(p) B_{\tau_{i_1}}(\phi_{-t_1}(p)) \cdots B_{\tau_{i_{\ell-1}}}(\phi_{-t_{\ell-1}}(p)) B_s(\phi_{-t_\ell}(p)).
	\end{equation}
	Thus
	\begin{equation}
		\| A_t(p) \| \, \| B_t(p)| \|^c \leq C_3 \left( \frac 1 2 \right)^\ell
	\end{equation}
	where 
	\begin{equation}
		C_3 = \max\{ \| A_s(q) \| \, \| B_s(q) \|^c : q\in \overline U, 0 \leq s \leq \tau_m\}.
	\end{equation}
	As \(t\to\infty\), we have \(\ell \to \infty\). Therefore \( \| A_t(p) \| \, \| B_t(p)| \|^c\to 0\) as \(t\to\infty\) and \(\sigma^s(p) \leq c\). We can then conclude that \(\sigma^s(p) \leq \sigma^s(K).\)
\end{proof}