% !TeX root = ../thesis.tex
\chapter{Existence of Kink-Like Travelling Wave Solutions}
\label{chp:existence}
\pagestyle{myheadings}

\section{Introduction}
% Talk about the problem
%Friesecke/Pego method. Why doesn't it work here?

%Use center manifold contruction
%%Usefule for finding small bounded solutions
%%Bifurcation parameter will be c<c^*

%Sketch of proof: 
%%Construct center manifold with epsilon as the bifurcation parameter
%%Show that (after an appropriate change of coordinates) that the epsilon = 0 system has a solution whose profile matches the kink solution of the  defocusing mKdV.
%% Then show that this solution persists for epsilon > 0 using Fenichel theory.
%% Later we reverse our change of coordinates to get an approximation of the solution on the original lattice formulation.

\section{Construction of Center Manifold}
%Change of coordinates to get into the form of an abstract ODE.

%The linear operator has a quadruple zero eigenvalue when mu = 1. 
%Write out the eigenvectors and their corresponding projection operators

%Go through the invariance argument to reduce to three eigenvectors (the invariance is a shift invariance on the original lattice).

% State the center manifold theorem at this point. Should discuss regularity of Phi and that it is in the null space of the spectral projection operators.

%Compute the taylor expansion of the center manifold function up to a certain order.
%Again make a change of coordinates to get a system in terms of epsilon.

\section{Existence of Heteroclinic Orbit}

%Take epsilon = 0. Then the system obviously has a heteroclinic orbit that is the profile for the kink solution for the mKdV. We need to show that 1) this solution exists and 2) it varys continuously wrt epsilon.

%Use Fenichel theory to get both of these

% Describe the overflowing invariant set. Compute the generalized Lyapunov coefficients. Use theorem (Wiggins book) to get an unstable manifold. The exact same result holds for the inflowing invariant set and its stable manifold.

%Show that the manifolds intersect transversally along the heteroclinic orbit at epsilon = 0

%