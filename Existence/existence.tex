% !TeX root = ../thesis.tex
\chapter{Existence of Kink-Like Traveling Wave Solutions}
\label{chp:existence}
\pagestyle{myheadings}

\section{Introduction}
% Talk about the problem
%Friesecke/Pego method. Why doesn't it work here?
The goal of this chapter is to show the existence of the travelling wave solution for the FPUT lattice and describe its profile. From formal calculations and the numerical experiments carried out in \cite{pace2019beta}, one expects that the travelling wave solution has a profile given by the kink solution to the mKdV; that is, for \(\phi(\xi) = \frac 1 {\sqrt 2} \tanh(\xi/\sqrt 2)\) we expect to have a travelling wave solution \(u\) such that 
\begin{equation}\label{formal-approximation}
	u_n(t) = \epsilon \varphi(\epsilon(n + ct)) +\mcO(\epsilon^3)
\end{equation}
when \(c\) is slightly smaller than \(V''(0) = 1\).

One would expect that methods used to find the soliton-like solution for the FPUT can also be applied to this case. Notably Friesecke and Pego showed in \cite{friesecke1999solitary} that there exists a solitary wave solution whose profile is described by the KdV soliton using a fixed-point argument. The argument relies on creating a map from \(H^1(\R)\) to itself using Fourier multipliers such that the fixed point of the map is the profile of the solitary wave. However, this argument does not extend to our case since the function \(\phi\) is not in a Sobolev space and its Fourier transform is defined only in a distributional sense. Due to this problem, we neglect the functional approach and focus on techniques from bifurcation theory. 


%Use center manifold contruction
%%Usefule for finding small bounded solutions
%%Bifurcation parameter will be c<c^*
One common technique for constructing travelling wave solutions to PDEs is by using the center manifold theorem. For PDEs of one spatial and one temporal variable, the strategy is to assume that the solution is a travelling wave (i.e.\ of the form \(f(x-ct)\)) to eliminate the derivative with respect to \(t\) and reduce the problem to an ODE with respect to the spatial variable \(x\). Finding bounded solutions of this ODE then results in travelling wave solutions of the PDE. The center manifold is an important tool for finding these solutions since (1) it is finite-dimensional, (2) can typically be approximated by Taylor series up to arbitrary order, and (3) contains all bounded solution. If a linear operator has an eigenvalue pass through the line \(\{\lambda\in\mathbb C: \Re \lambda = 0\}\) as a parameter \(\mu\) varies, then one typically has a center manifold containing small bounded parameterized by \(\mu\). Such a construction was carried out in \cite{iooss2000travelling}, in which the existence of several travelling wave solutions were proved. The bifurcation parameter in this paper was given in part by the wave speed. In fact, [Thm.\ 5]\cite{iooss2000travelling} shows the existence of a heteroclinic orbit on the center manifold when \(c\) is slightly smaller than \(1\). This heteroclinic orbit corresponds to the kink-like solution of the FPUT we are interested in. But no description of its wave profile was given, so obtaining an estimate of the form in \cref{formal-approximation} is still an open problem.


%Sketch of proof: 
%%Construct center manifold with epsilon as the bifurcation parameter
%%Show that (after an appropriate change of coordinates) that the epsilon = 0 system has a solution whose profile matches the kink solution of the  defocusing mKdV.
%% Then show that this solution persists for epsilon > 0 using Fenichel theory.
%% Later we reverse our change of coordinates to get an approximation of the solution on the original lattice formulation.
Our argument for getting such an estimate will proceed as follows. We first follow the procedure in \cite{iooss2000travelling}
to construct the center manifold parameterized by \(\epsilon\), making sure to explicitly compute the dynamics on the center manifold. Making a suitable change of variables, we look for small-amplitude, long-wavelength solutions for the FPUT on the center manifold and show that formally setting \(\epsilon = 0\) gives a solution related to the kink solution \(\phi\). Next we apply results from Fenichel theory to show that this solution persists for \(\epsilon> 0\). Lastly we convert our results back to the original formulation of the FPUT lattice and prove an estimate of the form \cref{formal-approximation}.

\section{Construction of Center Manifold}
%Change of coordinates to get into the form of an abstract ODE.
We follow the construction of the center manifold carried out in \cite{iooss2000travelling}. Recall that the equations for the FPUT lattice are given by
\begin{equation}
	\ddot x_n = V'(x_{n+1} - x_n) - V'(x_n - x_{n-1}), \quad n\in\Z.
\end{equation}
We assume that \(V(x) = \frac 12 x^2 - \frac 1 {24} x^4 + \mcO(x^5)\) near \(x=0\). We make the ansatz that 
\begin{equation}
	x_n(\tilde t) = x(n-c\tilde t),
\end{equation}
where the \(x(t)\) on the right is a function from \(\R\) to \(\R\). Hence \(x(t)\) must satisfy the advance-delay differential equation
\begin{equation}\label{advance-delay-equation}
	\ddot x (t) = \mu \Big(V'(x(t+1) - x(t)) - V(x(t) - x(t-1)) \Big)
\end{equation}
where \(\mu = c^{-2}\). Instead of working directly with \cref{advance-delay-equation}, we rewrite the equation as a first-order differential equation in a Banach space. \Cref{advance-delay-equation} cannot be written as a differential equation in a finite-dimensional phase space, and so we use a Banach space to represent a ``slice'' of the function on the interval \([t-1,t+1]\) for \(t\in\R\). We introduce a new variable \(v\in[-1,1]\) and functions \(X(t,v) = x(t+v)\). We use the notation \(\xi(t) = \dot x(t)\), \(\delta^1X(t,v) = X(t,1)\), and \(\delta^{-1} X(t,v) = X(t,-1)\). Then letting \(U(t) = (x(t), \xi(t), X(t,v))^T\) represent our solution, \cref{advance-delay-equation} can be written as follows:
\begin{equation}\label{first-order-abstract-ode}
	\partial_t U = L_\mu U + M_\mu (U)
\end{equation}
where \(L_\mu\) is the linear operator
\begin{equation}
	L_\mu = \begin{pmatrix}
		0 & 1 & 0\\
		-2\mu & 0 & \mu(\delta^1 + \delta^{-1}) \\
		0 & 0 & \partial_v
	\end{pmatrix}
\end{equation}
and 
\begin{equation}
	M_\mu(U) = \mu (0, g(\delta^1X -x) - g(x- \delta^{-1} X), 0)^T
\end{equation}
where we define \(g(x)= V'(x) - x\). We will also require that \(X(t,0) = x(t)\), so that \(X(t,v) = x(t+v)\) and solutions of \cref{first-order-abstract-ode} correspond with solutions of \cref{advance-delay-equation}. We introduce the following Banach spaces for \(U\):
\begin{equation}
	\begin{aligned}
		\bbH &= \R^2 \times C[-1,1] \\
		\bbD &= \{U \in \R^2 \times C^1[-1,1] \mid X(0) = x\}
	\end{aligned}
\end{equation}
where the spaces have the usual maximum norms. The operator \(L_\mu\) is continuous from \(\bbD\) to \(\bbH\). Assuming that \(g\in C^4(I)\) where \(I\) is an open neighborhood around \(0\), we have \(M_\mu \in C^4(\bbD,\bbD)\).

Note that \cref{first-order-abstract-ode} is not well-posed and solutions may not correspond with the requirement that \(X(t,0) = x(t)\). However, we can show that there is a center manifold which contains global solutions and lies in \(\bbD\), and so we will be able to extract the travelling wave solutions that we are interested in.

%The linear operator has a quadruple zero eigenvalue when mu = 1. 
%Write out the eigenvectors and their corresponding projection operators
As shown in \cite[Lem.\ 1]{iooss2000travelling}, when \(\mu = \mu_0:= 1\) (i.e.\ when \(c = \sqrt{V''(0)} = 1\)) the linear operator \(L_{\mu_0}\) has a quadruple zero eigenvalue with the rest of the spectrum bounded uniformly away from the imaginary axis. This allows for the construction of a four-dimensional center manifold. This construction is not carried out explicitly in \cite{iooss2000travelling}, but it follows similarly to the calculations carried out in \cite{iooss2000travelling2} which relies on results in \cite{vanderbauwhede1992center}.

The four-dimensional eigenspace for \(\lambda = 0\) is spanned by the following generalized eigenfunctions:
\begin{equation}
	\begin{aligned}
		&\zeta_0 = (1,0,1)^T  &\zeta_1 = (0,1,v)^T \\
		&\zeta_2 = (0, 0, \frac 12 v^2)^T & \zeta_3 = (0,0,\frac 1 6 v^3)^T
	\end{aligned}
\end{equation}
which satisfy
\begin{equation}
	\begin{aligned}
		L_{\mu_0} \zeta_0 &= 0 \\
		L_{\mu_0} \zeta_1 &= \zeta_0 \\
		L_{\mu_0} \zeta_2 &= \zeta_1 \\
		L_{\mu_0} \zeta_3 &= \zeta_2.
	\end{aligned}
\end{equation}
The spectral projection onto the eigenspace can be found using the Laurent expansion in \(\mcL(\bbH)\) near \(\lambda = 0\)
\begin{equation}
	(\lambda \mathrm I - L_{\mu_0} )^{-1} = \frac{D^3}{\lambda^4} + \frac{D^2}{\lambda^2} + \frac{D}{\lambda^2} + \frac P \lambda - \tilde L _{\mu_0} ^{-1} + \lambda \tilde L_{\mu_0} ^{-1} - \cdots
\end{equation}
where \(P\) is the spectral projection, \(D = L_{\mu_0} P\), and \(\tilde L _{\mu_0} ^{-1}\) is the pseudo-inverse of \(L_{\mu_0}\) on the subspace \((\mathrm I - P)\bbH\) (see \cite{kato2013perturbation}). The spectral projection satisfies
\begin{equation}
\begin{aligned}
	PW &= ((PW)_x, (PW)_\xi, (PW)_X)^T \\
	&= (PW)_x \zeta_0 + (DW)_x\zeta_1 + (D^2W)_x\zeta_2 + (D^3W)_x \zeta_3
\end{aligned}
\end{equation}
The projection can be computed by finding the resolvent \((\lambda \mathrm I - L_\mu)^{-1}\) and then determining the residue of a meromorphic function. The resolvent operator is straightforward to compute. For \(F=(f_0,f_1,F_2)^T\in\bbH\), we want to find \(U = (x,\xi, X)^T\in \bbD\) such that
\begin{equation}
	(\lambda\mathrm I - L_\mu) U = F.
\end{equation}
The operator on the left-hand side when \(N(\lambda ;\mu) \neq 0\) where 
\begin{equation}
	N(\lambda;\mu) = - \lambda^2 + 2\mu(\cosh \lambda - 1)
\end{equation}
and \(U\) is given by
\begin{align}
	x &= -[N(\lambda;\mu)]^{-1}(\lambda f_0 + f_1 + \mu\tilde f_\lambda) \\
	\xi &= -[N(\lambda;\mu)]^{-1} \Big( [\lambda^2 + N(\lambda;\mu)]f_0 + \lambda f_1 + \mu\lambda \tilde f_\lambda \Big) \\
	X(v) &= e^{\lambda v}x - \int_0^v e^{\lambda(v-s)} F_2(s)\, ds
\end{align}
with 
\begin{equation}
	\tilde f_\lambda = \int_0^1 [-e^{\lambda(1-s)} F_2(s) + e^{-\lambda (1-s)} F_2(-s)]\, ds.
\end{equation}
Hence, the projection can be computed by standard techniques. For instance, note that 
\begin{equation}
	(PF)_x = \mathrm{Res}((\lambda \mathrm I - L_{\mu_0} ^{-1} F)_x, 0) = \mathrm{Res}(-[N(\lambda;\mu)]^{-1}(\lambda f_0 + f_1 + \mu\tilde f_\lambda), 0).
\end{equation}
For fixed \(F\in\bbH\), the last can be found by finding the residue of a meromorphic function in \(\bbC\). Proceeding in this way, we can get 
\begin{align}
	(PF)_x &= \frac 2 5 \Bigg(  f_0 - \int_0^1[(1-s) - 5(1-s)^3][F_2(s) + F_2(-s)]\, ds \Bigg) \\
	(DF)_x &= (PF)_\xi = \frac 2 5 \Bigg(  f_1 - \int_0^1[1 - 15(1-s)^2][F_2(s) - F_2(-s)]\, ds \Bigg) \\
	(D^2F)_x &= (DF)_\xi = -12 \Bigg( f_0 - \int_0^1(1-s)) [F_2(s) + F_2(-s)]\, ds \Bigg) \\
	(D^3F)_x &= (D^2 F)_\xi = -12 \Bigg(f_1 - \int_0^1 [F_2(s) - F_2(-s)]\, ds \Bigg).
\end{align}
We denote by \(\zeta_j^*\) the linear continuous forms on \(\bbH\) given for any \(F\in\bbH\) by
\begin{equation}
	\begin{aligned}
		\zeta_0^*(F)&= (PF)_x \\
		\zeta_1^*(F) &= (DF)_x  = \zeta_0^*(L_{\mu_0} F) \\
		\zeta_2^*(F) &= (D^2F)_x  \\
		\zeta_3^*(F) &= (D^3F)_x 
	\end{aligned}
\end{equation}
and we have that
\begin{equation}
	\zeta_k^*(\zeta_j) = \delta_{kj} \quad k,j = 0, 1, 2, 3
\end{equation}
where \(\delta_{kj}\) is the Kronecker delta.

%Go through the invariance argument to reduce to three eigenvectors (the invariance is a shift invariance on the original lattice).
At this point we could start to compute the four-dimensional center manifold parameterized by \(\mu\), but we can do a further simplification. Note that \cref{first-order-abstract-ode} is invariant under 
\begin{equation}
	U \mapsto U + q \zeta_0, \quad \forall q \in \R
\end{equation}
which corresponds to the shift invariance of \cref{advance-delay-equation}. This invariance allows us to reduce the center manifold to a three-dimensional manifold. We first decompose \(U \in \bbH\) as follows:
\begin{equation}
	U = W + q \zeta_0, \quad \zeta_0^*(W) = 0.
\end{equation}
Denote by \(\bbH_1\) to codimension-one subspace of \(\bbH\) where \(\zeta_0^*(W) = 0\), and similarly define \(\bbD_1\). Then the system in \cref{first-order-abstract-ode} becomes
\begin{align}
	\frac{dq}{dt} &= \zeta_0^*(L_{\mu} W) = \zeta_0^*(L_{\mu_0} W) = \zeta_1^*(W) \\
	\frac{d W}{dt} &= \widehat{L}_\mu W + M_\mu(W) \label{reduced-first-order-system}
\end{align}
where \(\widehat{L}_\mu W = L_\mu W - \zeta_1^*(W)\zeta_0\). The operator \(\widehat L_{\mu_0}\) acting on \(\bbH_1\) has the same spectrum as \(L_{\mu_0}\) except that \(0\) is now a triple eigenvalue instead of a quadruple eigenvalue. One can check that
\begin{equation}\label{reduced-eigenfunction}
	\widehat{L}_{\mu_0} \zeta_1 = 0, \quad \widehat{L}_{\mu_0}\zeta_2 = \zeta_1, \quad \widehat{L}_{\mu_0}\zeta_3 = \zeta_2, \quad \zeta_3^*(\widehat{L}_{\mu_0}W) = 0.
\end{equation}

% State the center manifold theorem at this point. Should discuss regularity of Phi and that it is in the null space of the spectral projection operators.
Hence we have a three-dimensional center manifold on which solutions are given by
\begin{equation}\label{W-center-manifold}
	W = A \zeta_1 + B\zeta_2 + C\zeta_3 +\Phi_\mu(A,B,C).
\end{equation}
Here \(\Phi_\mu\) takes values in \(\bbD_1\). Note that this implies solutions on the center manifold correspond with solutions of \cref{advance-delay-equation}, as desired. We also have that \(\Phi_\mu\) (1) has the same regularity as \(V'\), (2) satisfies \(\zeta_k^*(\Phi_\mu) = 0\) for \(k=1,2,3\),  and (3) is at least quadratic in its arguments.

It is at this point that our discussion diverges from the work in \cite{iooss2000travelling}. From this point, Iooss uses the reversibility of the vector field and results from normal form theory to study the existence of homoclinic, heteroclinic, and periodic solutions on the center manifold. However, since there is an unspecified change of coordinates, the results in \cite{iooss2000travelling} do not give quantitative estimates but rather qualitative descriptions of the solutions. For our purposes though, we would like to compare the profile of the travelling wave solutions and compare it to the mKdV kink solution, and so we must proceed differently. We shall instead compute the Taylor expansion of \(\Phi_\mu\) up to a certain order and get an explicit representation of the center manifold (up to some specified error).


%Compute the taylor expansion of the center manifold function up to a certain order.
%Again make a change of coordinates to get a system in terms of epsilon.
We assume that \(\Phi_\mu\) can be written as a Taylor series in \(A,B,C\), and \(\mu\):
\begin{equation}\label{phi-taylor-series}
	\Phi_\mu(A,B,C) = \sum_{i,j,k,\ell} (\mu - 1)^\ell A^i B^j C^k \Phi^{(\ell)}_{ijk} 
\end{equation}
Note that the \(\mu\) terms are centered at \(\mu_0 = 1\). We will only need to compute up to some of the cubic terms, so we do not need \(\Phi_\mu\) is analytic as suggested by \cref{phi-taylor-series}. In fact \(\Phi_\mu \in C^4\) in a neighborhood of \((\mu, A, B, C) = (1,0,0,0)\) is sufficient and is guaranteed by the regularity we assumed for \(V'\) and \(g\).

It is useful to compute \(\widehat L_\mu\) applied to each eigenvector:
\begin{align}
	\widehat L_\mu \zeta_1 &= 0 \\
	\widehat L_\mu \zeta_2 &= \zeta _1  + (\mu-1)\begin{bmatrix}0 \\ 1 \\ 0 \end{bmatrix} \\
	\widehat L_\mu \zeta_3 &= \zeta_2.
\end{align}
Note that these calculations agree with \cref{reduced-eigenfunction} when \(\mu\) is equal to \(\mu_0 = 1\). Now plugging \cref{W-center-manifold} into \cref{reduced-first-order-system} gives
\begin{multline}
	\dot A \zeta_1 + \dot B \zeta_2 + \dot C \zeta_3 + D\Phi_\mu(A,B,C) \begin{bmatrix} \dot A \\ \dot B \\ \dot C \end{bmatrix} = \\
	B \zeta_1 + B(\mu-1) \begin{bmatrix}0 \\ 1 \\ 0 \end{bmatrix} + C \zeta_3 + L_{\mu_0} \Phi_\mu(A,B,C) \\
	+ \left(2(1-\mu) \Phi_\mu^x + (\mu-1)(\delta^1 \Phi_\mu^X + \delta^{-1} \Phi_\mu^X)\right) \begin{bmatrix}0 \\ 1 \\ 0 \end{bmatrix} \\
	+ \mu\left( g\big(A + \frac 1 2 B + \frac 1 6 C + (\delta^1 \Phi_\mu^X -\Phi_\mu^x)\big) - g\big(A - \frac 1 2 B + \frac 1 6 C + (\Phi_\mu^x - \delta^{-1} \Phi_\mu^X)\big) \right) \begin{bmatrix}0 \\ 1 \\ 0 \end{bmatrix}
\end{multline}
where we represent the components of \(\Phi_\mu\) by \((\Phi_\mu^x, \Phi_\mu^\xi, \Phi_\mu^X)^T\). Now we can group the \(\zeta_1\), \(\zeta_2\) and \(\zeta_3\) together -- as well as the remaining terms -- to get a system of differential equations on the center manifold:
\begin{align}
	\dot A =& B + \frac 2 5 \Big[ \cdots \Big] \label{dotA}\\
	\dot B =& C \label{dotB}\\
	\dot C =& -12 \Big[\cdots \Big] \label{dotC} \\
	D\Phi_\mu&(A,B,C) \begin{bmatrix} \dot A \\ \dot B \\ \dot C \end{bmatrix} = L_{\mu_0} \Phi_\mu + \Big[ \cdots \Big] \begin{bmatrix} 0 \\ \frac 35 \\ 2v^3 - \frac 2 5 v \end{bmatrix} \label{perp}.
\end{align}
The \(\cdots\) within the brackets are given the following expression
\begin{equation}
\begin{aligned}
	&B (\mu-1) + 2(1-\mu) \Phi_\mu^x + (\mu-1)(\delta^1 \Phi_\mu^X + \delta^{-1} \Phi_\mu^X) \\
	&+  \mu\left( g\big(A + \frac 1 2 B + \frac 1 6 C + (\delta^1 \Phi_\mu^X -\Phi_\mu^x)\big) - g\big(A - \frac 1 2 B + \frac 1 6 C + (\Phi_\mu^x - \delta^{-1} \Phi_\mu^X)\big) \right),
\end{aligned}
\end{equation}
which we abridged to improve legibility. Now using the expression for the derivatives in \cref{dotA,dotB,dotC} and plugging into \cref{perp} gives the following:
\begin{equation}\label{final-eq}
	\frac{\partial\Phi}{\partial A}\left( B + \frac 2 5 [\cdots]\right) + \frac{\partial\Phi}{\partial B}C + \frac{\partial\Phi}{\partial C}\left(-12 [\cdots]\right) = L_{\mu_0} \Phi_\mu + [\cdots] \begin{bmatrix} 0 \\ \frac 35 \\ 2v^3 - \frac 2 5 v \end{bmatrix}.
\end{equation}
We will now assume \(\Phi_\mu\) has the form given in \cref{phi-taylor-series}. From the center manifold theorem, we have that the first-order terms and the terms quadratic in just \(A\), \(B\), and \(C\) are zero. Thus we start by first computing the second-order terms where \(\ell = 1\). We get the following set of equations:
\begin{align}
	\Phi^{(1)}_{100}  & = L_{\mu_0} \Phi^{(1)}_{010} + \begin{bmatrix} 0 \\ \frac 35 \\ 2v^3 - \frac 2 5 v \end{bmatrix} \label{phi-1-010} \\
	\Phi^{(1)}_{010}  & = L_{\mu_0} \Phi^{(1)}_{001}  \label{phi-1-001}\\
	0 & = L_{\mu_0} \Phi^{(1)}_{100} \label{phi-1-100}
\end{align}
\Cref{phi-1-100} can be solved by noting that \(\zeta_0\) is the only zero eigenfunction for \(L_{\mu_0}\) and \(\zeta_0^*(\Phi_{100}^{(1)}) = 0\) since \(\Phi_\mu\) takes values in \(\bbD_1\), thus \(\Phi_{100}^{(1)} = 0\). Then \cref{phi-1-010} is reduced to 
\begin{equation}
	0 =  L_{\mu_0} \Phi^{(1)}_{010} + \begin{bmatrix} 0 \\ \frac 35 \\ 2v^3 - \frac 2 5 v \end{bmatrix},
\end{equation}
which can be solved by integrating to get 
\begin{equation}
	\Phi_{010}^{(1)} = \begin{bmatrix}
		0 \\ 0 \\ - \frac 1 2 v^4 + \frac 1 5 v^2
	\end{bmatrix} + k  \zeta_0
\end{equation}
for some \(k\in \R\). Imposing the constraint that \(\zeta_0^*(\Phi_{010}^{(1)}) = 0\) gives us that \[k = -13/2100.\] Similarly integrating \cref{phi-1-001} gives
\begin{equation}
	\Phi_{001}^{(1)} = \begin{bmatrix}
		0 \\ 0 \\ - \frac 1 {10} v^5 + \frac 1 {15} v^3
	\end{bmatrix} + k  \zeta_1
\end{equation}
with the same value of \(k\).

One can similarly compute the cubic coefficient \(\Phi_{300}^{(0)}\) and get that
\begin{equation}
	\Phi_{300}^{(0)} = 0.
\end{equation}

We will not need to compute any of the other coefficients. As we will soon see, after a change of variables they end up being in the higher order terms to be neglected. Before proceeding, we will need a new parameterization for the center manifold. We let \(\epsilon > 0\) correspond with the amplitude of our travelling wave solution, and look to write \(\mu = c^{-2}\) in terms of \(\epsilon\). As seen in \cite{iooss2000travelling}, the heteroclinic orbits on the center manifold will only exists for \(c^2\) slightly less than \(1\). Based on some formal calculations, it appears \(c^2 = 1 - \epsilon^2/12\) will be the correct scaling. This will be borne out by the coming calculations. Thus we have \[\mu-1 = c^{-2} - 1 = \frac 1 {1-\epsilon^2/12} -1 = \frac {\epsilon^2}{12} +\mcO(\epsilon^4).\] Since we are looking for \(\epsilon\)-amplitude waves with wavelength of order \(\epsilon^{-1}\), we make the following change of variables:
\begin{equation}
	A(t) = \epsilon \underline A (\epsilon t), \quad B(t) = \epsilon^2 \underline B(\epsilon t), \quad \epsilon^3 \underline C(\epsilon t).
\end{equation}
Then the equations of motion on the center manifold become
\begin{equation}\label{eqns-center-manifold}
\begin{aligned}
	\underline A ' &= \underline B + \mathcal O(\epsilon) \\
	\underline B ' &= \underline C \\
	\underline C ' &= - \underline B + 6 \underline A^2 \underline B + \mathcal O(\epsilon).
\end{aligned}
\end{equation}
Here the \(\mcO(\epsilon)\) represents functions that are at least \(C^4\) in \(\epsilon\), \(\underline A\), \(\underline B\), and \(\underline C\) and can be bounded by a constant times \(\epsilon\) when we are on bounded domains and \(\epsilon >0\) sufficiently small. Since we will be looking for bounded solutions on the center manifold, these terms can be controlled. We may upgrade this to \(\mcO(\epsilon^2)\) if we additionally have \(V(x) = \frac 1 2 x^2 - \frac 1 {24} x^4 + \mcO(x^6)\) as \(x\to 0\).

We shall consider two cases going:
\begin{enumerate}[label = (\arabic*)]
	\item \(V(x) = \frac 12 x^2 - \frac 1 {24} x^4 + \mcO(\epsilon^5)\)
	\item \(V(x) = \frac 12 x^2 - \frac 1 {24} x^4 + \mcO(\epsilon^6)\)
\end{enumerate}
For (1), we have that the flow on the center manifold is given by
\begin{equation}\label{epsilon-flow}
	\begin{aligned}
		\underline A ' &= \underline B + \epsilon F_1(\underline A, \underline B, \underline C;\epsilon) \\
		\underline B ' &= \underline C \\
		\underline C ' &= - \underline B + 6 \underline A^2 \underline B + \epsilon G_1(\underline A, \underline B, \underline C;\epsilon) \\
		\epsilon' &= 0
	\end{aligned}
\end{equation}
where \(F_1\) and \(G_1\) will be \(C^4\) for \(\epsilon > 0\) and \(\mcO(1)\) as \(\epsilon \to 0\). The additional equation \(\epsilon' = 0\) is added so that we may use \(\epsilon\) as an additional coordinate in our results. Note that this will not change the flow on the center manifold since \(\epsilon\) remains fixed. For (2), we parameterize based on \(\eta = \epsilon^2\) and the flow is now given by
\begin{equation}\label{eta-flow}
	\begin{aligned}
		\underline A ' &= \underline B + \eta F_2(\underline A, \underline B, \underline C;\sqrt \eta) \\
		\underline B ' &= \underline C \\
		\underline C ' &= - \underline B + 6 \underline A^2 \underline B + \eta G_2(\underline A, \underline B, \underline C;\sqrt\eta) \\
		\eta' &= 0
	\end{aligned}
\end{equation}
where \(F_2\) and \(G_2\) will be \(C^4\) for \(\eta > 0\) and \(\mcO(1)\) as \(\eta \to 0\). Reparameterizing to \(\eta\) will ultimately allow us to improve our error from \(\mcO(\epsilon)\) to \(\mcO(\epsilon^2)\). The systems can be extended to \(C^1\) flows for negative values of the parameters: for instance we make the replacement
\begin{equation}
\begin{aligned}
	\epsilon F_1(\underline A, \underline B, \underline C; \epsilon) &\rightarrow \epsilon F_1(\underline A, \underline B, \underline C; |\epsilon|) \\
	\epsilon G_1(\underline A, \underline B, \underline C; \epsilon) &\rightarrow \epsilon G_1(\underline A, \underline B, \underline C; |\epsilon|)
\end{aligned}
\end{equation}
to get \cref{epsilon-flow} is \(C^1\) for (possibly negative) \(\epsilon\) near zero. A similar replacement of \(\sqrt \eta \rightarrow \sqrt{|\eta|}\) makes \cref{eta-flow} \(C^1\) for \(\eta\) near zero. 

The arguments for the persistence of heteroclinic orbits is similar for \cref{epsilon-flow,eta-flow}, so we will focus first on the former system and note where the results differ for the latter system.
\section{Existence of Heteroclinic Orbit}
%Take epsilon = 0. Then the system obviously has a heteroclinic orbit that is the profile for the kink solution for the mKdV. We need to show that 1) this solution exists and 2) it varys continuously wrt epsilon.

%Use Fenichel theory to get both of these

% Describe the overflowing invariant set. Compute the generalized Lyapunov coefficients. Use theorem (Wiggins book) to get an unstable manifold. The exact same result holds for the inflowing invariant set and its stable manifold.

%Show that the manifolds intersect transversally along the heteroclinic orbit at epsilon = 0
At this point, our goal is to show the existence of a heteroclinic orbit for \cref{eqns-center-manifold} for \(\epsilon>0\) sufficiently small and to get estimates of the solution. One might expect that the flow on the center manifold for \(\epsilon>0\) small is well approximated by formally setting \(\epsilon = 0\). Indeed, if we let \(\epsilon = 0\), then the ODEs in \cref{eqns-center-manifold} become equivalent to the third-order differential equation
\begin{equation}
	\underline A ''' + \underline A'  - 6 \underline A^2 \underline A' = 0 
\end{equation}
which has the solution 
\begin{equation}
	\underline A(s) = \frac 1 {\sqrt 2} \tanh\left(\frac s {\sqrt 2}\right).
\end{equation}
This solution is the profile for the kink solution of the defocusing mKdV, \(\phi\). This represents a heteroclinic orbit for the system of ODEs since \((\underline A(s), \underline B(s), \underline C(s)) \to (\pm1/\sqrt 2, 0, 0)\) as \(s\to\pm \infty.\) One might expect that for \(\epsilon > 0\) that there is also a heteroclinic orbit that is close to the above solution. Thus we want to show that the heteroclinic orbit at \(\epsilon = 0\) persists for small perturbations of \(\epsilon\), and we want to get estimates of these orbits relative to \(\epsilon\). To get these results, we apply Fenichel theory. Review \cref{sec:fenichel_theory} for the relevant results that will be used.

The idea behind the proof is to show that there is an overflowing invariant set with an unstable manifold and a corresponding inflowing invariant set with a stable manifold. We then show that at \(\epsilon = 0\) these manifolds intersect transversally at a point, and that this intersection is given by the above heteroclinic orbit. From there, we show that this intersection is preserved for \(\epsilon > 0\) and the heteroclinic orbit remains \(\mcO(\epsilon)\) or \(\mcO(\epsilon^2)\) to the original orbit.

\subsection{The Unstable and Stable Manifolds}

We first must find the appropriate overflowing invariant set.\footnote{We need also to find the inflowing invariant set, but we can rely on the symmetry of \cref{eqns-center-manifold} to get this. In fact, we will regularly rely on the symmetry of the flow to get many of the results for the inflowing invariant set after working it out for the overflowing invariant set.} From the heteroclinic orbit found for \(\epsilon = 0\), we know that \((A,B,C, \epsilon) = (-1/\sqrt 2 , 0 ,0, 0)\) should be one point in the set. In fact, for fixed \(\epsilon> 0\) we have that multiples of \(\zeta_1\) are fixed points for \cref{reduced-first-order-system}. From the center manifold theorem in \cite{vanderbauwhede1992center}, bounded solutions sufficiently close to the origin will lie exactly on the center manifold. Thus for \(\epsilon>0\) sufficiently close to zero, any closed interval on the \(\underline A\)-axis is composed entirely of fixed points on the center manifold. We will choose \(\epsilon_0> 0\) small enough such that for \(\epsilon \in (0,\epsilon_0]\) the \(\underline A\)-axis from \([-1,1]\) is composed entirely of fixed points. 



If we fix a small \(\delta > 0\) and set \(A_{-\infty} = - 1/ \sqrt 2\), then 
\begin{equation}
	\overline M = \{(\underline A, 0, 0, \epsilon) \in \R^4: |(\underline A - A_{-\infty},  \epsilon)| \leq \delta\}
\end{equation}
is a smooth manifold with boundary that is invariant under the flow in \cref{epsilon-flow}. In fact, \(\overline M\) consists exclusively of fixed points of the flow.

To get apply \cref{unstable-manifold-fenichel} and get an unstable manifold for \(\overline M\) we need that 
\begin{enumerate}[label=(\roman*)]
	\item \(\overline M\) is overflowing invariant, and 
	\item the generalized Lyapunov-type numbers on \(\overline M\) satisfy the inequalities in \cref{unstable-manifold-fenichel}.
\end{enumerate}
As written, \(\overline M\) is \emph{not} an overflowing invariant manifold. However, a common trick in Fenichel is to adjust the flow on the boundary of an invariant manifold so that it becomes overflowing invariant (see \cite[\S 6.3]{wiggins1994normally}). This will alter the behavior of our dynamical system at the boundary, but elsewhere the dynamics will remain the same. For our case, we may adjust the flow near the boundary \(|(\underline A - A_{-\infty},  \epsilon)| = \delta\) to get \(\overline M\) is overflowing invariant, but this will not affect the dynamics near the heteroclinic orbit. Thus we can still talk about the existence of the heteroclinic orbit in the unaltered system. This adjustment will need to be done in a way to not greatly affect the generalized Lyapunov-type numbers. For now, we set aside point (i) and address (ii), which is more straightforward. 

Since \(\overline M\) consists only of fixed points, the generalized Lyapunov-type numbers can be computed using the linearization of the flow. Note that since each \((\underline A, 0, 0,\epsilon) \in \overline M\) is a fixed point, we have that
\begin{equation}
	\begin{aligned}
		F_1(\underline A, 0, 0, \epsilon) &= 0 \\
		G_1(\underline A, 0, 0, \epsilon) &= 0 \\
	\end{aligned}
\end{equation}
and the partial derivatives of \(F_1\) and \(G_1\) with respect to \(\underline A\) or \(\epsilon\) will be zero. Thus at a point \((\underline A, 0, 0, \epsilon) \in\overline M\), the linearization of the flow is given by
\begin{equation}
	\begin{bmatrix}
		0 & 1 + \epsilon \frac{\partial F_1}{\partial \underline B}(\underline A, 0, 0; \epsilon) & \epsilon \frac{\partial F_1}{\partial \underline C}(\underline A, 0, 0;\epsilon) & 0 \\
		0 & 0 & 1 & 0 \\
		0 & 6\underline A ^2 - 1 + \epsilon \frac{\partial G_1}{\partial \underline B}(\underline A, 0, 0; \epsilon) & \epsilon \frac{\partial G_1}{\partial \underline C}(\underline A, 0, 0; \epsilon) & 0 \\
		0 & 0 & 0 & 0
	\end{bmatrix}.
\end{equation}
The tangent space at \(p\in M\) is given by \(T_p M = \mathrm{span}\{(1,0,0,0), (0,0,0,1)\}\). The vector bundles \(N^u\) and \(N^s\) will be defined as the unstable and stable subspaces of each fixed point, respectively. That these vector bundles are invariant under the flow and continuous follows immediately from their definition. The two eigenvalues \(\lambda_1,\lambda_2=0\) correspond with the flow tangent to the manifold. Fixing \(\underline A = A_{-\infty}\) and \(\epsilon = 0\), the other eigenvalues are \(\lambda_{3,4} = \pm \sqrt{6A_{-\infty}^2-1}\), which correspond with the flow along the vector bundles \(N^u\) and \(N^s\), respectively. There at \(p_0 =(A_{-\infty}, 0, 0, 0)\)  we have the generalized Lyapunov-type numbers given by
\begin{equation}
	\lambda^u(p_0) = \nu^s(p_0) = \exp\big(-\sqrt{6A_{-\infty}^2-1}\big), \quad \sigma^s(p_0) = 0.
\end{equation}
To have a \(C^1\) unstable manifold, we are required to have \(\lambda^u(p), \nu^s(p), \sigma^s(p)  < 1\)  for each point \(p\in M\). By the continuity of eigenvalues, we can guarantee this by choosing \(\delta\) small enough.

Then condition (ii) is satisfied. Now we want to show that we can alter near \(\partial M\) so that (i) is also satisfied without causing (ii) to become invalid. We first introduce a \(C^\infty\) bump function, \(\chi:[0, \infty) \to \R\), such that 
\begin{enumerate}[label = (\arabic*)]
	\item \(0\leq \chi(r) \leq 1\) for \(r \in[0,\infty)\)
	\item \(\chi(r) = 0\) when \(r\in [0, \delta -\sigma]\)
	\item \(\chi(r) = 1\) when \(r\in [\delta - \frac \sigma 2 , \delta + \frac \sigma 2]\)
	\item \(\chi(r) = 0\) when \(r\in [\delta +\sigma,\infty)\)
\end{enumerate}
where \(\sigma > 0\) will be a parameter that we can make as small as necessary. We then alter the vector field in \cref{epsilon-flow} by setting
\begin{equation}
\begin{aligned}
	\underline A ' &= \underline B + \epsilon F_1(\underline A, \underline B, \underline C;\epsilon) + \chi( |(\underline A - A_{-\infty},  \epsilon)|) \cdot (\underline A - A_{-\infty}) \\
	\underline B' &= \underline C \\
	\underline C ' &= - \underline B + 6 \underline A^2 \underline B + \epsilon G_1(\underline A, \underline B, \underline C;\epsilon) \\
	\epsilon' &= \chi( |(\underline A - A_{-\infty},  \epsilon)|) \cdot \epsilon.
\end{aligned}
\end{equation}
This change keeps the flow \(C^1\) and makes \(\overline M\) an overflowing invariant vector field. However, a couple things need to be checked before applying \cref{unstable-manifold-fenichel}: the vector bundles \(N^u\) and \(N^s\) must be defined on \(\chi \neq 0\) and the generalized Lyapunov-type numbers must satisfy the necessary inequalities.

The extension of the normal vector bundles is somewhat technical, but otherwise straightforward. We need \(N^u\) and \(N^s\) invariant under the flow and continuous. In particular, if \(p\) is a point in \(M\) where \(\chi \neq 0\) and \(\xi \in N^u_p\), then \(D\phi_t(p) \xi \in N^u_{\phi_t(p)}\) for all \(t\) such that \(\phi_t(p) \in M\). A similar result should hold for \(N^s\). We also need that the vector bundles are continuous with respect to \(p\). Continuity relies on showing that we can assign the vector bundles in a way so that if \(\phi_t(p) \to p'\) as \(t\to-\infty\) then \(N^u_{\phi_t(p)}\) approaches \(N^u_{p'}\). This can be done and the details are carried out in \cref{condition-verification}.

For the generalized Lyapunov-type numbers, it can be shown that the values on the altered region of \(M\) can be bounded by those on the unaltered region. More generally, we have the following result.
\begin{restatable}{prop}{lyapunovbound}
	Let \(K \subset M\) be a compact set. If \(p \in M\) such that \(\phi_{-t}(p) \to K\) as \(t\to\infty\), then 
\begin{enumerate}[label=(\roman*)]
	\item \(\lambda^u(p) \leq \lambda^u(K)\), 
	\item \(\nu^s(p) \leq \nu^s(K)\), and 
	\item if \(\nu^s(K) < 1\), then \(\sigma^s(p) \leq \sigma^s(K)\).
\end{enumerate}
\end{restatable}
The proof is give in \cref{condition-verification} and follows similarly to the arguments found in \cite{dieci1997lyapunov}.

We can therefore conclude that \(W^u_{\mathrm{loc}}(\overline M)\) exists. If we set \(A_\infty = 1/ \sqrt 2\), then an analogous argument holds for showing that
\begin{equation}
	\overline N = \{(\underline A, 0, 0, \epsilon) \in \R^4: |(\underline A - A_{\infty},  \epsilon)| \leq \delta\}
\end{equation}
has a \emph{stable} manifold, \(W^s_{\mathrm{loc}}(\overline N)\).

\subsection{Transversal intersection at $\epsilon=0$}
To show a heteroclinic orbit exists for \(\epsilon > 0\), we first show that stable and unstable manifolds described above have a transverse intersection at \(\epsilon = 0\). This intersection then persists for perturbations in \(\epsilon\) (since the manifolds are \(C^1\) with respect to \(\epsilon\)) and thus implies the existence of the heteroclinic orbit.

The heteroclinic orbit at \(\epsilon = 0\) can be found explicitly. The dynamics (away from where we modified the vector field) are given by
\begin{equation}\label{ep-zero-system}
	\begin{aligned}
		\underline A ' &= \underline B \\
		\underline B' &= \underline C \\
		\underline C' &= - \underline B + 6 \underline A^2 \underline B.
	\end{aligned}
\end{equation}
The system of ODEs in \cref{ep-zero-system} has two invariants:
\begin{align}
	I_1(\underline A, \underline B, \underline C) &= \underline C + \underline A - 2 \underline A^3 \\
	I_2(\underline A, \underline B, \underline C) &= \frac 1 2 \underline B^2 + \frac 12 \underline A^2 - \frac 1 2 \underline A^4 - \underline A I_1(\underline A, \underline B, \underline C).
\end{align}
We then look for solutions on the manifolds given by
\begin{equation}
	I_1(A_{-\infty}, 0, 0) = 0 \quad \text{ and } \quad I_2(A_{-\infty}, 0, 0) = \frac 1 8.
\end{equation}
The above equations and the fact that \(\underline B = \underline A'\) gives us that \(\underline A\) must satisfy the following first order ODE:
\begin{equation}
	(\underline A')^2 = \left(\frac  1 2 - \underline A^2\right)^2,
\end{equation}
which can be solved by separation of variables. The solutions are thus
\begin{equation}
	\underline A(s) = \pm \frac{1}{\sqrt 2} \tanh\left(\frac s {\sqrt 2} \right)
\end{equation}
up to a shift in the variable \(s\). One can check that these are solutions of \cref{ep-zero-system} (taking \(\underline B = \underline A'\) and \(\underline C = \underline A''\)) and define two heteroclinic orbits: one traveling from \(A_{-\infty}\) to \(A_\infty\) and one traveling from \(A_{\infty}\) to \(A_{-\infty}\). The solution corresponding with the choice of \(+\) also lies inside the manifolds \(W^u_{\mathrm{loc}}(\overline M)\) and \(W^s_{\mathrm{loc}}(\overline N)\) since it converges to \(\overline M\) and \(\overline N\) as \(s\to - \infty \) and \(s\to +\infty\), respectively. This does not imply the local manifolds intersect since they are only defined in a neighborhood of \(\overline M\) and \(\overline N\), but we may extend these manifolds under the flow so that they both contain the point \((\epsilon , \underline A, \underline B, \underline C) = (0, 0, 1/2, 0)\) and thus intersect. We shall refer to the manifolds extended under the flow by \(\mathcal M_\epsilon\) and \(\mathcal N_\epsilon\). These extended manifolds are still \(C^1\) with respect to the parameter \(\epsilon\).

Now the goal is to demonstrate that this intersection is transverse. That is for \(p = (0,1/2,0)\) we want to show that \(T_p\mathcal M_0 + T_p \mathcal N_0 = T_p \R^3\). One can explicitly compute each of the tangent spaces at \(p\) and show they span \(T_p\R^3\). This is done by finding the intersection each of these manifolds make with the \(\underline B\underline C\)-plane. Similar to the construction of the heteroclinic orbit, we find the orbit which approaches some asymptotic value on the \(\underline A\)-axis near \(A_{-\infty}\) or \(A_{\infty}\) and find where it intersect the \(\underline B\underline C\)-plane. These orbits lie on the stable and unstable manifolds, and so this shows how the manifolds intersect the plane.

Take \(\omega\) to be a point near \(A_{-\infty} = - 1/\sqrt 2\). The orbit that approaches \((\omega, 0, 0)\) in backwards time lies on the intersection of 
\begin{equation}
	\begin{aligned}
		I_1(\underline A, \underline B, \underline C) &= I_1(\omega, 0, 0) = \omega - 2\omega^3 \\
		I_2(\underline A, \underline B, \underline C) &= I_2(\omega, 0, 0) = - \frac 1 2 \omega^2 + \frac 3 2 \omega^4.
	\end{aligned}
\end{equation}
Setting \(\underline A = 0\), we can find that \(\mathcal M_0\) hits the \(\underline B \underline C\)- plane at 
\begin{equation}
	\mu(\omega) = (0, |\omega| \sqrt{3\omega^2 -1 }, \omega - 2\omega^3)
\end{equation}
for \(\omega\) close to \(A_{-\infty}\). In particular, we see that if \(\mu(A_{-\infty}) = p\). Identical reasoning gives that \(\mathcal N_0\) intersects the plane at 
\begin{equation}\label{intersection-with-plane}
	\nu(\alpha) = (0, |\alpha| \sqrt{3\alpha^2 -1 }, \alpha - 2\alpha^3)
\end{equation}
where \(\alpha\) is near \(A_\infty = 1 / \sqrt 2\) and \(\nu (A_{\infty} ) = p\).

The derivative of the heteroclinic orbit is given by \((1/2, 0, -1/2)\), and this vector lies in both \(T_p\mathcal M_0\) and \(T_p \mathcal N_0\). Since \(\mu(\omega) \in \mathcal M_0\) for \(\omega\) near \(A_{-\infty}\), we have that 
\begin{equation}
	\mu'(A_{-\infty}) = \left( 0, 1, \frac  1 {\sqrt 2} \right) \in T_p \mathcal M_0.
\end{equation}
Similarly,
\begin{equation}
	\nu'(A_{-\infty}) = \left( 0, 1, \frac  {-1} {\sqrt 2} \right) \in T_p \mathcal N_0.
\end{equation}
Therefore
\begin{equation}
	T_p \mathcal M_0 + T_p \mathcal N_0 = \mathrm{span} \left\{ \left(\frac 1 2 , 0, \frac{-1} 2\right),  \left( 0, 1, \frac  1 {\sqrt 2} \right) ,  \left( 0, 1, \frac  {-1} {\sqrt 2} \right) \right\} = T_p\R^3
\end{equation}
and the intersection is transverse. This implies that there is a heteroclinic orbit on the intersection of \(\mathcal M_\epsilon\) and \(\mathcal N_\epsilon\) for \(\epsilon > 0\).