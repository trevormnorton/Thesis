% !TeX root = ../thesis.tex

\chapter{Introduction}

\section{A brief history of the FPUT lattice}

This thesis is concerned with the Fermi-Pasta-Ulam-Tsingou (FPUT) lattice, an infinite set of differential equations posed on the lattice \(\Z\) with a nearest-neighbor interaction. The equations can be written as
\begin{equation}\label{fput-lattice-odes}
	\ddot x_n = V'(x_{n+1}- x_n) - V'(x_n - x_{n-1}), \quad n\in \Z
\end{equation}
where \(V(x)\) is the potential. Another common way to write the equations is using the strain variables, \(u_n = x_ n - x_{n-1}\), in which case the equations become
\begin{equation}\label{fput-lattice-equations-strain-variables}
	\ddot{u}_n = V'(u_{n+1}) - 2 V'(u_n) + V'(u_{n-1}), \quad n \in \Z.
\end{equation}
The lattice first became of interest when it was used to model the thermalization process in a solid. In \cite{fermi1955studies}, researchers numerically computed solutions of the FPUT on a large, finite lattice with a nonlinear potential given by
\begin{equation}
	V(x) = \frac 1 2 x^2 + \frac 1 6 x^3 + \text{h.o.t.},
\end{equation}
up to a rescaling. The FPUT with this potential is typically referred to as the \(\alpha\)-FPUT chain. The initial condition for the system had its energy concentrated in its first Fourier mode, and it was believed that the nonlinear coupling would cause an equipartition of energy across all the modes after a sufficient amount of time. Surprisingly, after a certain period the system made a near-recurrence to its initial condition; around 97\% of the energy returned to the first mode. The cause for this recurrence was unknown, and the phenomenon was labeled a paradox. 

Progress was made in  \cite{zabusky1965interaction} by looking at the continuum limit of the lattice. If one imagines moving the points of the lattice \(\Z\) closer together, then the real line could be used as an approximation in the limit. Looking at this limiting case, Zubusky and Kruskal were able to show that solutions of the FPUT could be modeled by solutions of the Korteweg-de Vries (KdV) equation. The KdV is a dispersive partial differential equation given by 
\begin{equation}\label{kdv}
	u_t  - 6 u u_x  + u_{xxx}= 0
\end{equation}
and is commonly used as a model for shallow water waves. Importantly, it is an example of an integrable PDE and has soliton solutions. A soliton is a solitary wave solution with the ``properties of a particle''. For instance, two solitons can pass through each other without a change of shape. The existence of soliton solutions is characteristic of many integable PDEs.

The soliton solutions help explain the behavior of the \(\alpha\)-FPUT and its near-recurrence. Using the KdV as the continuum limit, the initial condition decomposes into a sum of solitons moving at different speeds. These solitons move up and down the finite interval, passing through each other, until they nearly all return to their initial position, which results in the recurrence. 

This argument was mainly heuristic, but the idea was later made rigorous. In particular, Friesecke and Pego show that there exists a solitary wave solution to the \(\alpha\)-FPUT chain whose profile can be approximated by the soliton of the KdV and also demonstrated stability of the solution on the lattice \cite{friesecke1999solitary,friesecke2002solitary,friesecke2003solitary,friesecke2004solitary}. Asymptotic stability in front of the solitary wave in the space \(\ell^2\) was shown by Mizumachi in \cite{mizumachi2009asymptotic} and later expanded to the \(N\)-solitary wave case in \cite{mizumachi2013asymptotic}. Further work shows how the KdV can be used more widely as a modulation equation for small-amplitude, long-wavelength solutions of the \(\alpha\)-FPUT lattice. Wayne and Schneider showed that the KdV can be used to approximate counter-propagating waves for long periods of time \cite{schneider2000counter} of the order \(\epsilon^{-3}\), where \(\epsilon > 0\) is the amplitude of the solutions. This idea is expanded in \cite{khan2017long}, where it is demonstrated that the KdV can approximate solutions of the FPUT for time of the order \(\epsilon^{-3}\log |\epsilon|\) which is then used to comment on the meta-stability of traveling wave solutions on the lattice. 

\section{The \(\beta\)-FPUT chain}

The \(\beta\)-FPUT chain is the system we get when the potential equals
\begin{equation}
	V(x) = \frac 12 x^2 \pm \frac 1 {24}x^4 + \text{h.o.t.},
\end{equation}
up to a rescaling. As opposed to the \(\alpha\)-FPUT chain where the cubic term can be made positive after rescaling, the sign of quartic term in the \(\beta\)-FPUT chain cannot be changed by rescaling, and its value affects the behavior of the system. In either case, this system  experiences the recurrences as first demonstrated in the \(\alpha\)-FPUT chain. There is also a continuum limit to the lattice equation given by the modified Korteweg- de Vries equation (mKdV), written as
\begin{equation}
	u_t \pm 6 u^2 u_x + u_{xxx} = 0.
\end{equation}
The \(\pm\) in the above equation is determined by the corresponding sign in the quartic term of \(V(x)\). The choice of \(+6u^2u_x\) gives the focusing mKdV, and the choice of \(-6u^2u_x\) gives the defocusing mKdV. Similar to solitons in the KdV, the defocusing mKdV has traveling wave solutions known as kinks. Kinks have profiles which approach non-zero values at \(\pm \infty\); for example, \(\frac 1 {\sqrt 2} \tanh(\frac 1 {\sqrt 2} (x+t))\) is such a solution.

In \cite{pace2019beta}, the recurrence in the \(\beta\)-FPUT chain was numerically studied. When looking at the lattice when \(V(x) = \frac  12 x^2 - \frac 1 {24}x^4\), the recurrence seemed to be driven by kink-like solutions of the FPUT in an analagous way to the role solitary wave solutions played in the \(\alpha\)-FPUT recurrence. This suggests that these kink-like solutions are an important object of study and can help explain the recurrence occurring in the \(\beta\)-FPUT chain. However, there has been relatively little research into these kink-like solutions or the defocusing mKdV as a continuum limit for the lattice equations. 

The research problems can roughly be divided into three parts: 1) deriving the existence of a kink-like solution and estimating its wave profile; 2) approximating small-amplitude, long-wavelength solutions of the FPUT by using the defocusing mKdV as a modulation equation; and 3) determining the stability of the kink-like solution on the lattice. While these areas are related, each comes with its own tools and challenges that need to be considered separately. 


\section{Research challenges and layout for thesis}

The first challenge is to determine the existence of a traveling wave solution whose profile is approximated by the defocusing mKdV kink solution. One notable technique for finding traveling wave solutions is \emph{spatial dynamics}. For PDEs and other dynamical systems, the idea is to assume a special form of the solution so that the derivatives in time disappear; hence the spatial variable can be thought of as the variable in time. This usually allows one to reduce the problem to finding solutions of an ODE, which is often much simpler. For example, spatial dynamics can be used to derive the explicit form of the kink solutions of the defocusing mKdV. Assume we want a solution of 
\begin{equation}\label{intro-defocusing-mkdv}
	u_t - 6u^2u_x + u_{xxx} = 0
\end{equation}
of the form
\begin{equation}
	u(x,t) = \varphi(x+ct) 
\end{equation}
with \(c > 0\). Substituting the ansatz into \cref{intro-defocusing-mkdv} gives
\begin{equation}
	c \varphi '(\xi) - 6\varphi(\xi)^2 \varphi'(\xi) + \varphi'''(\xi) = 0,
\end{equation}
which is an ODE in the variable \(\xi=x+ct\). Taking an antiderivative on the left-hand side and setting that to zero gives
\begin{equation}
	c\varphi  - 2 \varphi^3  + \varphi'' = 0
\end{equation}
The above ODE can be solved by noting that it is Hamiltonian with \(H(q,p) = \frac 1 2 p^2 + \frac c 2 q^2 - \frac 12 q^4\) and the non-zero equilibria of the system occur when \((\varphi, \varphi') = (\pm \frac c {\sqrt 2}, 0)\). Then finding the solutions along the level set \(H(\pm \frac c {\sqrt 2}, 0) = \frac {c^2} 8\) gives 
\begin{equation}
	\varphi(\xi) = \pm \sqrt{\frac c  2} \tanh\left( \sqrt{\frac c 2} \xi\right)
\end{equation}
and the (increasing) kink solutions of \cref{intro-defocusing-mkdv} are 
\begin{equation}\label{kink-soln-defocusing-mkdv}
	\varphi_c(x+ct) = \sqrt{\frac c 2} \tanh\left(\sqrt{\frac c 2} (x+ct) \right).
\end{equation}

Spatial dynamics is less useful in the context of lattices since the spatial domain in that case is discrete. Trying something along the same lines for equations on the lattice would result in a difference equations, which are more difficult to extract explicit solutions from. Thus other methods must be used. In \cite{friesecke1999solitary}, the authors set up a fixed point problem in \(H^1(\R)\) for the profile of the traveling wave by using a Fourier transform. Hence they were able to get existence of the solution and describe the profile of the wave. A challenge in our particular case is that kink solutions do not lie in any nice Sobolev space like \(H^1(\R)\), so defining Fourier transforms becomes difficult.

A more robust strategy for finding traveling wave solutions can be found in \cite{iooss2000travelling}. Here the author uses a center manifold construction to find traveling wave solutions on the FPUT lattice. Center manifolds are a useful tool for finding these special solutions, since they contain slowly growing/decaying solutions of a dynamical system. A close reading of \cite{iooss2000travelling} shows that kink-like solutions of the \(\beta\)-FPUT exist. However, the profiles of these solutions are not analyzed with respect to the profile of the kink solutions of the mKdV. In this thesis, Iooss' work is enhanced to get explicit descriptions of the profile for these kink-like solutions and show that they are well-approximated by the mKdV kink solutions.

The next challenge to consider is more generally considering the defocusing mKdV as a modulation equation. This would extend the results found in \cite{schneider2000counter,khan2017long} to our \(\beta\)-FPUT lattice. For these papers and other similar approximation results, the strategy is to use an ansatz for the approximate form of the solution and show that the residual remains suitably small for some period of time. Typically this involves the careful choice of an energy function  to bound the residual followed by a Gr\"onwall-type argument. In the case of \cite{schneider2000counter}, the ansatz involves two counter-propagating solutions and so some coupling between equations of motion occurs. This is dealt with by assuming some localization of the ansatz solution; if the two counter-propagating waves are localized in space, then the interference that occurs is limited and the error due to the coupling can be bounded globally in time.\footnote{An alternative approach is to use the theory of dispersive PDEs to control this coupling, as seen in \cite{hong2021korteweg}.} In the case of a kink solution, this localization assumption cannot be met since it does not decay to zero in space. This assumption can be replaced with one that forces the ansatz solution to approach its limit at infinity suitably fast. This allows for an analogous argument where the coupling can be controlled globally in time. 

In \cite{khan2017long}, the KdV approximation is shown to hold for a longer time scale, and so one can then comment on the metastability of solitary wave solutions in the FPUT. Showing a similar approximation holds for the kink-like solution would also allow conclusions of the metastability of the solution from the stability of kink solution of the defocusing mKdV.

The final challenge would be to characterize the stability of the kink-like solution globally in time. The previous approximation result only demonstrates the stability for long but finite time, so one would like to extend this to \emph{all} time. For traveling wave solutions, there are several notions of stability that can be considered. Spectral stability occurs when the spectrum of the operator after linearizing around the traveling wave solution is contained in the left-half plane of \(\mathbb C\). The Evans function, \(D(\lambda)\), is a common tool for determining spectral stability in PDEs. If \(\lambda\) is an eigenvalue for the linear operator, then \(D(\lambda) = 0\) and so the Evans function can be used to locate eigenvalues. Linear stability is a related notion, which states that solutions for the linearized equation decay to zero in time. Typically spectral stability implies linear stability. A stronger form of stability for traveling wave solutions is orbital stability. This is analogous to Lyapunov stability in ODEs; a solution which is near a traveling wave solution will stay close to a traveling wave. Since we typically have a family of wave solutions parameterized by displacement and/or wave speed, we allow for small changes in these parameters. One can control this directly by setting up modulation equations for these parameters so that the approximation remains as close as possible to the family of traveling wave solutions. The strongest form of stability is asymptotic stability, where if you start near a traveling wave solution then you will \emph{converge} to a traveling wave solution.

Stability in infinite-dimensional systems has the added difficulty of determining the appropriate norm for which stability holds. While in finite dimensions every norm is equivalent and so stability results remain the same despite the norm, for infinite dimensions we may have stability hold in one space but not in another.  For example, it is common for traveling wave solutions of dispersive PDEs to not have spectral stability in \(L^2(\R)\), since the spectrum of the linear operator touches the axis \(i \R\), but moving to an exponentially weighted space can shift the spectrum to the left and establish spectral stability.

Linear stability is often the first step toward showing orbital or asymptotic stability. For example, it was necessary to show linear stability of the solitary wave solution of the \(\alpha\)-FPUT in \cite{friesecke2003solitary,friesecke2004solitary} before asymptotic stability in the full system could be shown. Linear stability is shown by developing a Floquet theory for shifts on the lattice. In \cite{mizumachi2013asymptotic}, we see a different way to derive linear stability. This is accomplished by taking a Fourier transform of the linear equation on the lattice and then decomposing the solution into low- and high-frequency parts. The high-frequency parts can be controlled separately. The low-frequency parts are controlled by comparing them to the linearized KdV solution. A similar argument can be made in the defocusing mKdV case. This suggests the following course of action for determining stability of the kink-like solution: show that the kink solution is linearly stable in the defocusing mKdV; use this result to show that the kink-like solution is also linearly stable; and finally use linear stability of the kink-like solution to get asymptotic stability.

This thesis can be divided into three main chapters. \Cref{chp:existence} focuses on showing the existence of the kink-like solution on the lattice. The center manifold reduction from \cite{iooss2000travelling} is followed to get explicit dynamics on the manifold. From there, it is shown that the limiting dynamics as we send the parameter \(\epsilon\) to \(0\) are given by the mKdV. Fenichel theory can then be applied to show the kink solution of the mKdV approximates solutions on the center manifold, and thus the kink-like solution on the lattice. \Cref{chp:long-time-stability} focuses on more generally using the defocusing  mKdV as a modulation equation for small-amplitude, long-wavelength solutions of the \(\beta\)-FPUT. Particular attention is paid toward ensuring estimates hold for solutions with non-zero limits at infinity. It is shown that counter-propagating waves can be approximated for periods of time of order \(\mcO(\epsilon^{-3})\), and for single wave solutions this time can be extended to \(\mcO(\epsilon^{-3} \log|\epsilon|)\). Finally, \cref{chp:stability} discusses the problem of stability for the kink-like solution. It is shown that the kink solution of the defocusing mKdV is linearly stable. This is done by first describing the spectrum of the linear operator obtained from linearizing around the kink solution, and then those results are used to show that there is decay in the semigroup generated by the linear operator. Next we discuss how to show linear stability of the kink-like solution in the FPUT and sketch and argument that would show this. \Cref{condition-verification} details some of the basic Fenichel theory needed in \cref{chp:existence} and also proves some specialized results. \Cref{lemma-appendix} is devoted to proving technical lemmas that come up in the main chapters.